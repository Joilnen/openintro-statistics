\chapter{Introdução aos dados}
\label{introductionToData}

%\begin{tipBox}{\tipBoxTitle[Chapter Goal:]{Thinking about data}
%Understand basics about data organization, data types, numerical summaries of data, graphical summaries of data, and foundational techniques for data collection. We begin and end the chapter with case studies.}
%\end{tipBox}

Cientistas procuram responder questões usando métodos rigorosos e cuidadosas observações. Estas observações -- coletando de notas de campo, pesquisas, e experimentos -- formam a espinha dorsal da investigação estatística e são chamados \term{dados}. Estatística é o estudo de como melhor serem coletados, analisados, e desenhar conclusões dos dados. É útil pormos a estatística no contexto da processo geral de investigação:
\begin{enumerate}
\setlength{\itemsep}{0mm}
\item Identifique uma questão ou problema.
\item Colete dados relevantes sobre o tópico.
\item Analise os dados.
\item Elabore uma conclusão.
%\item Make decisions based on the conclusion.
\end{enumerate}
A estatística foca nos 2-4 objetivos, rigorosamente, e eficientemente. Com foco nestas três premissas: Qual a melhor forma de coletar os dados ? Como deveria ser analisado ? E o que podemos inferir desta análise ?

Os assuntos científicos são diversos como as questões por eles levantados. No entanto, muitas destas investigações podem ser abordadas com um peno número de técnicas de coleta de dados, ferramentas de análise, e conceito fundamental de inferência estatística. Este capítulo traz uma introdução deste e outros temas que serão abordados neste livro. Introduzimos os princípios básicos de cada tópico e estudaremos algumas ferramentas. Iremos também trazer questões de outras áreas não tipicamente associadas com a ciência, mas com um benefício importante no estudo da estatística.

\section[Estudo de caso: usando stents para prevenir infartos]{Estudo de caso: usando stents para prevenir infartos \sectionvideohref{youtube-nEHFF1ADpWE&list=PLkIselvEzpM6pZ76FD3NoCvvgkj_p-dE8}}
\label{basicExampleOfStentsAndStrokes}

\index{data!stroke|(}

Sessão~\ref{basicExampleOfStentsAndStrokes} introduz um desafio clássico em estatística: avaliando a eficiência de um tratamento médico. Abordado nesta sessão, e também no resto do capítulo com mais detalhes, e revisado ao fim do texto. O plano agora e simplesmente entendermos o papel que a estatística pode ter na prática.

Nesta sessão iremos considerar um experimento que estuda a eficácia dos stents no tratamento de paciente com risco de infarto.\footnote{Chimowitz MI, Lynn MJ, Derdeyn CP, et al. 2011. Stenting versus Aggressive Medical Therapy for Intracranial Arterial Stenosis. New England Journal of Medicine 365:993-1003. \oiRedirect{textbook-nejm_stent_study}{www.nejm.org/doi/full/10.1056/NEJMoa1105335}. NY Times article reporting on the study: \oiRedirect{textbook-nytimes_stent_study}{www.nytimes.com/2011/09/08/health/research/08stent.html}.} Stents são dispositivos instalados em vasos sanguíneos que ajudam a recuperação do paciente depois de um infarto e reduz o risco de um próximo infarto ou mesmo a morte. Muitos médicos acreditam em um benefício similar para paciente com risco de infarto. A principal questão a ser respondida é:
\begin{quote}
Stents reduzem o risco de infarto ?
\end{quote}

Os pesquisadores tem feito esta pergunta com relação a 451 pacientes com risco de infarto. Cada um dos pacientes voluntários foi aleatoriamente incluído em um de dois grupos:
\begin{itemize}
\item[]\termsub{Grupo de tratamento}{grupo de tratamento group}. Paciente neste grupo recebem um stent e medicação. A medicação tem efeito terapêutico, mudanças nos fatores de rico e ajuda na mudança do estilo de vidan.
\item[]\termsub{Grupo de controle}{grupo de controle}. Paciente no grupo de controle recebem o mesmo medicamento do primeiro grupo, mas não recebem o stents.
\end{itemize}
Os pesquisadores aleatoriamente incluem 224 pacientes para o grupo de tratamento e 227 pacientes para o grupo de controle. Neste estudo, o grupo de control provê o ponto de referência com o qual mensuramos o impácto do tratamento com stents no grupo de tratamento.
Os pesquisadores estudaram os efeitos do stents em dois pontos: 30~dias após o início do experimento e 365~dias passados. Os resultados de 5 paciente estão resumidos na tabela Table~\ref{stentStudyResultsDF}. Os resultados foram marcados com ``infarto'' ou ``nenhum evento'', representando que o paciente nao teve infarto até o final da pesquisa.

\begin{table}[h]
\centering
\begin{tabular}{l ccc}
\hline
Paciente	&	grupo	&	0-30 dias 	&	0-365 dias \\
\hline
1		&	tratamento &	nenhum evento &	nenhum evento \\
2		&	tratamento &	infarto & infarto \\
3		&	tratamento &	nenhum evento & nenhum evento \\
$\vdots$	&	$\vdots$	  &	$\vdots$ \\
450	&	controle &	nenhum evento &	nenhum evento \\
451	&	controle &	nenhum evento &	nenhum evento \\
\hline
\end{tabular}
\caption{Resultados para 5 paciente no estudo sobre stents.}
\label{stentStudyResultsDF}
% trmt <- c(rep('trmt', 224), rep('control', 227)); outcome30 <- c(rep(c('event', 'no_event'), c(33, 191)), rep(c('event', 'no_event'), c(13, 214))); outcome365 <- c(rep(c('event', 'no_event'), c(33, 191)), rep(c('event', 'no_event'), c(13, 214)))
\end{table}
%% parei#aqui
Considerando os dados de cada paciente individualmente levaria muito tempo, um caminho não viável para responder a questão original da pesquisa. Portanto, fazendo-se uma análise de dados estátisticos nos permite considerar os dados de todos os pacientes de uma única vez. Table~\ref{stentStudyResults} resume os dados brutos de uma maneira mais útil. Nesta tabela, nós rapidamente vemos o que acontece no estudo completo. Identificamos o número de paciente no grupo de tratamento que tiveram infarto dentro de 30 dias, olhando a esquerda da table no cruzamento da coluna Infarto com a linha Tratamento: 33

\begin{table}[h]
\centering
\begin{tabular}{l cc c cc}
& \multicolumn{2}{c}{0-30 dias} &\hspace{5mm}\ & \multicolumn{2}{c}{0-365 dias} \\
  \cline{2-3} \cline{5-6}
	& 	infarto 	& nenhum evento && 	infarto 	& nenhum evento \\
  \hline
tratamento 	& 33		& 191	&&	45 	& 179 \\
controle 		& 13		& 214	&& 	28	& 199 \\
  \hline
Total				& 46		& 405	&&	73	& 378 \\
  \hline
\end{tabular}
\caption{Estatística  para  o estudo sobre stent.}
\label{stentStudyResults}
\end{table}

\begin{exercise}
Dos 224 paciente no grupo de tratamento, 45 tiveram infarto no primeiro ano. Usando estes dois números, calcule a proporção de pacientes no grupo de tratamento que tiveram um
    infarto no primeiro ano. (Nota: respostas para todos os exercícios práticos nas notas de página.)\footnote{A proporção dos 224 paciente que tiveram infarto no período de 365 dias: $45/224 = 0.20$.}
\end{exercise}

Podemos calcular uma síntese desta tabela. Ou seja, um ``sumário estatístico'' é um número único que resume uma extensa quantidade de dados. \footnote{Formalmente, um sumário estatístico é um valor calculado dos dados. Alguns sumários estátisticos são mais úteis que outros. Por enquanto os resultados primários do estudo após 1~ano pode ser descrito por dois sumários estatísticos: a proporção de pessoas que tiveram infarto em tratamento pelas do grupo de tratamento.}

\begin{itemize}
\setlength{\itemsep}{0mm}
\item[] Proporção de quem teve infarto no grupo de tratamento (stent): $45/224 = 0.20 = 20\%$.
\item[] Proporção de quem teve infarto no grupo de controle: $28/227 = 0.12 = 12\%$.
\end{itemize}

Estes dois sumários são úteis olhando as diferenças entre os grupos, nos causa surpresa : 8\% dos pacientes no grupo de tratamento tiveram infarto! Isto é importante por duas rasões. Primeiro, isto contraria o esperado pelos médicos, que acretitavam que o stents, poderiam \emph{reduzir} a taxa de infartos. Segundo, isto produz a seguinte questão estatística: os dados mostram uma ``real'' diferença entre os grupos?

Sobre esta segunda questão surgida. Suponha que você jogue uma moeda 100 vezes. Enquanto a chance de dar cara é 50\%, provavelmente nós não observaremos esta probabilidade exatamente. Este tipo de flutuação é parte de quase qualquer tipo de geração de dados. É possível que a diferença de 8\% no estudo sobre stents, é uma variação normal. De qualquer forma, diferenças maiores observadas (para um particular quantidade de dados), é menos confiável que seja apenas uma flutuação natural. Então devemos nos perguntar: esta diferença é grande o bastante para desprezarmos a possibilidade de flutuação natural na produção dos dados ?

Enquanto ainda não temos ferramentas estatísticas para abordarmos esta questão, nós poderíamos concluir por estas análises: existem evidências para desaconselhar o uso de stents neste estudo de pacientes cardíacos.

\textbf{Seja cuidadoso:} não generalize os resultados deste estudo para todos os pacientes e todos os tipos de stents. Este estudo verificou pacientes com algumas variáveis específicas, voluntários para serem parte do estudo e podem não serem representativos no universo de pacientes cardíacos. Temos também muitos tipos de stents e este estudo considerou apenas o auto-expansível Wingspan stent (Boston Scientific). De qualquer forma, este estudo é uma lição importante para nós: pode ainda nos surpreender.

\index{data!stroke|)}

\section[Dados básicos]{Dados básicos \sectionvideohref{youtube-Mjif8PTgzUs&list=PLkIselvEzpM6pZ76FD3NoCvvgkj_p-dE8}}
\label{dataBasics}

Efetiva representação e descrição dos dados é o primeiro passo na maioria das análises. Esta sessão introduz uma estrutura para organização dos dados bem como alguma terminologia que será utilizada ao longo deste livro.

\subsection{Observações, variáveis, e matriz de dados}

\index{data!email50|(}

Table~\ref{email50DF} mostra colunas 1, 2, 3, e 50 do conjunto de dados relativos a 50 emails recebidos durante o começo de 2012. Estas observações serão referenciadas como \data{email50} data set, e são uma amostra aleatória de um extenso conjunto quem pode ser visto aqui Section~\ref{categoricalData}.

Cada linha na tabela representa um único email ou \term{caso}.\footnote{Um caso pode ser chamado também \term{unidade de observação} ou uma \term{unidade observacional}.} As colunas representam características, chamadas \termsub{variables}{variable}, para cada um dos emails. Por exemplo, a primeira linha representa email 1, o qual não é spam, contém 21, 705 caracteres, 551 quebras de linha, está escrito em no formato HTML, e contém apenas números pequenos.

Na prática, é especialmente importante perguntar e deixar claro questões importantes sobre os aspectos dos dados a serem estendidos. De agora em diante, será sempre importante estarmos certo do que cada variável significa e as unidade em que foram medidas. Descrições de todas as cinco variáveis dos emails dadas na tabela Table~\ref{email50Variables}.
\begin{table}[t]
\centering
\begin{tabular}{cc ccc c}
  \hline
 & \var{spam} & \var{num\_\hspace{0.3mm}carac} & \var{quebra\_\hspace{0.3mm}linha} & \var{formato} & \var{número} \\
  \hline
1 & não & 21,705 & 551 & html & pequeno \\
  2 & não & 7,011 & 183 & html & grande \\
  3 & sim & 631 & 28 & text & none \\
$\vdots$ & $\vdots$ & $\vdots$ & $\vdots$ & $\vdots$ & $\vdots$ \\
  50 & não & 15,829 & 242 & html & pequeno \\
   \hline
\end{tabular}
\caption{Quatro linhas de \data{email50} matriz de dados.}
\label{email50DF}
\end{table}
% library(openintro); library(xtable); data(email50); email50[c(1,2,3,50),c("spam", "num_char", "line_breaks", "format", "number")]; xtable(email50[c(1,2,3,50),c("spam", "num_char", "line_breaks", "format", "number")], digits=0)


\begin{table}[t]
\centering\small
\begin{tabular}{lp{10.5cm}}
\hline
{\bf variáveis} & {\bf descrição} \\
\hline
\var{spam} & Especifica qualquer email spam \\
\var{num\_\hspace{0.3mm}carac} & O número de caracteres do email \\
\var{quebra\_\hspace{0.3mm}linha} & O número de quebra de linhas no email (não incluído continuação na linha seguinte)   \\
\var{formato} & Indica se o email contém formato especial, como negrito, tabelas ou links, os quais indicam que o email está em conteúdo formatado como HTML   \\
    \var{número} & Indica que o email não está contido em nenhum número, num pequeno número (menos que 1 milhão), ou um número maior   \\
\hline
\end{tabular}
\caption{Descrições das variáveis \data{email50} conjunto de dados.\textC{\vspace{-3.5mm}}}
\label{email50Variables}
\end{table}

\index{data!email50|)}

Os dados na tabela Table~\ref{email50DF} representa um term{matriz de dados}, o qual é um jeito comum de organizar os dados. Cada linha da matriz de dados corresponde a únic caso, e cada coluna corresponde a uma variável. Uma matriz de dados para o estudo sobre infartos mostrado na Section~\ref{basicExampleOfStentsAndStrokes} é mostrado na Table~\vref{stentStudyResultsDF}, onde os caso foram paciente e tínhamos três variáveis registrado para cada paciente.

Matriz de dados é um jeito conveniente para armazenar os dados. Se outro dado, ou caso é adicionado ao conjunto de dados, uma adicional linha pode ser facilmente adicionada. Similarmente, outra coluna pode ser adicionada para uma nova variável.

\index{data!county|(}

\begin{exercise}
    Nós consideramos um conjunto de dados publicamente disponível o que resume informações dos 3.143 municípios dos Estados Unidos, chamaremos isto \data{county} conjunto de dados. Este conjunto de dados inclui informação sobre cada município: nome, estado onde se localiza, sua população entre 2000 e 2010, gasto percápito, taxa de pobreza, e cinco características adicionais. Como devem estes dados serem organizado numa matriz de dados? Lembre: olhe para as notas na base das páginas para verificas as respostas dos exercícios no texto.\footnote{Cada município pode ser visto como um case, e há onze informações específicas para cada município vistas no case. Uma tabela com 3.143 linhas e 11 colunas contendo os dados, onde cada linha representa um município e cada coluna uma informação específica.}
\end{exercise}

\noindent Sete linhas do \data{county} conjunto de dados mostrados na Table~\ref{countyDF}, e as variáveis resumidas na Table~\ref{countyVariables}. Estes dados foram coletados do US Cesnsus website.\footnote{\oiRedirect{textbook-census_quick_facts}{quickfacts.census.gov/qfd/index.html}}

\begin{landscape}
\begin{table}
\centering\small
\begin{tabular}{ccc ccc ccc ccc}
\hline
& \var{nome} & \var{estado} & \var{pop2000} & \var{pop2010} &
   \var{percátpito\_\hspace{0.3mm}gasto} & \var{tx. de pobreza} & \var{casa própria} & \var{multi} &
   \var{renda} & \var{renda\_\hspace{0.3mm}média} & \var{poluição\_disp\hspace{0.3mm}} \\
\hline
  1 & Autauga & AL & 43671 & 54571 & 6.068 & 10.6 & 77.5 & 7.2 & 24568 & 53255 & não \\
  2 & Baldwin & AL & 140415 & 182265 & 6.140 & 12.2 & 76.7 & 22.6 & 26469 & 50147 & não \\
  3 & Barbour & AL & 29038 & 27457 & 8.752 & 25.0 & 68.0 & 11.1 & 15875 & 33219 & não \\
  4 & Bibb & AL & 20826 & 22915 & 7.122 & 12.6 & 82.9 & 6.6 & 19918 & 41770 & não \\
  5 & Blount & AL & 51024 & 57322 & 5.131 & 13.4 & 82.0 & 3.7 & 21070 & 45549 & não \\
  $\vdots$ & $\vdots$ & $\vdots$ & $\vdots$ & $\vdots$ & $\vdots$ & $\vdots$ & $\vdots$ & $\vdots$ & $\vdots$ & $\vdots$ & $\vdots$ \\
  3142 & Washakie & WY & 8289 & 8533 & 8.714 & 5.6 & 70.9 & 10.0 & 28557 & 48379 & não \\
  3143 & Weston & WY & 6644 & 7208 & 6.695 & 7.9 & 77.9 & 6.5 & 28463 & 53853 & não \\
\hline
\end{tabular}
\caption{Sete linhas do \data{county} conjunto de dados.}
\label{countyDF}
% library(openintro); county <- countyComplete[,c("name", "state", "pop2000", "pop2010", "fed_spending", "poverty", "home_ownership", "housing_multi_unit", "per_capita_income", "median_household_income")]; colnames(county) <- c("name", "state", "pop2000", "pop2010", "fed_spend", "poverty", "homeownership", "multiunit", "income", "med_income"); county$fed_spend <- county$fed_spend / county$pop2010

% library(openintro); library(xtable); cc <- countyComplete; xtable(cc[c(1:5,nrow(cc) - (1:0)), c("name", "state", "pop2000", "pop2010", "fed_spend", "poverty", "homeownership", "multiunit", "income", "med_income")], digits=1)
\end{table}

\begin{table}
\centering\small
\begin{tabular}{lp{11cm}}
\hline
{\bf variável} & {\bf descrição} \\
\hline
\var{name} & Nome do município \\
\var{state} & Estado onde o município se localiza (incluindo o Distrito de Columbia) \\
\var{pop2000} & População in 2000 \\
\var{pop2010} & População in 2010 \\
\var{percápito\_\hspace{0.3mm}gasto} & Federal gasto perçapito \\
\var{tx. de pobreza}  &  Percentual da população na pobreza \\
\var{casa própria}  &  Percentual da população que possiu e mora em casa própria (crianças morando com pais que vivem em casas próprias) \\
\var{multi}  &  Percentual de moradias que são várias unidades juntas (ex: apartamentos) \\
\var{renda} & Renda per capita \\
\var{renda\_\hspace{0.3mm}média} & Mediano da renda da moradia, considerado o total da renda dos ocupantes da casa com idade igual ou superior a 15 anos \\
\var{poluição\_disp\hspace{0.3mm}}  &  Tipo de município amplo e com satisfatória dispersão da poluição até o fim de 2011, valores possíveis, um dos três:
			\resp{não}, \resp{parcial}, or \resp{ok},
			onde \resp{ok}  significa efetiva disperção da poluição \\
\hline
\end{tabular}
\centering
\caption{Variáveis e suas descrições para \data{county} conjunto de dados.}
\label{countyVariables}
\end{table}
\end{landscape}

\subsection{Tipos de variáveis}
\label{variableTypes}

Examine o \var{gasto\_\hspace{0.3mm}federal}, \var{pop2010}, \var{estado}, and \var{disp\_\hspace{0.3mm}poluição} variáveis no \data{county} conjunto de dados. Cada uma das variáveis é hierarquicamente diferente das outras com características singulares.

Primeiro consideramos \var{gasto\_\hspace{0.3mm}federal}, o qual foi dito ser variável \term{numérica} desde que isto pode ser uma longa faixa de valores, e isto é importante ser adicionado, subtraído, ou ter as médias destes valores. De outra forma, nós poderíamos classifica uma variável código de área telefónica como numérica, e a média, soma, diferença não ter significado claro algum.

A variável \var{pop2010} também é numérica, apesar de parecer um pouco diferente de \var{gasto\_\hspace{0.3mm}federal}. Esta variável que conta a população apenas pode assumir valores não negativos (\resp{0}, \resp{1}, resp{2}, ...). Por esta razão, a variável da população é dita \term{discreta} desde que isto pode ser unicamente numera, e valores com saltos. De outra maneira, o gasto federal percápito é dita uma variável contínua

As variáveis \var{state} pode assumir até 51 valores como, alem de Washington, DC:  \resp{AL}, ..., and \resp{WY}. Respondem como categorias, \var{state} então é chamada variável categórica, e possíveis valores são chamados nível das variáveis.

\begin{figure}
\centering
\includegraphics[width=0.57\textwidth]{ch_intro_to_data/figures/variables/variables}
\caption{Breakdown of variables into their respective types.}
\label{variables}
\end{figure}

Finalmente, considere a variável \var{smoking\_\hspace{0.3mm}ban}, a qual descreve o tipo de county-wide somokin band e guarda valores \resp{none}, \resp{parcia}, ou \resp{comprehensive} em cada município. Esta variável parece ser híbrida: é categórica, mas os níveis tem uma ordenação natural. Uma variável como esta propriedade é chamada \term{ordinal}, enquanto uma regular variavel categórica sem este tipo especial é conhecida \term{nominal}. Para simplificar as análises, qualquer variável neste livro será tratada como variáveis categóricas.

\begin{example}{Dados onde são coletados estatísticas sobre os estudantes do curso. Três variáveis contém os seguintes itens: número de irmãos, altura, se o estudante já teve seus dados previamente verificados. Classifica cada uma das variáveis como numérica, discreta numérica ou categórica.}
O número de irmãos e altura do estudante são representados por variáveis numéricas. O número de siblings é um contador discreto, as variáveis de altura são contínuas, então numérica contínua, e a última variável classifica os estudantes em duas categorias os que já tiveram e os quem não tiveram suas estatísticas auferidas, portanto, uma variável categórica.
\end{example}

\begin{exercise} \index{data!stroke}
Considere as variáveis \var{group} e \var{outcome} (em 30 days) do estudo sobre uso do stent na Section~\ref{basicExampleOfStentsAndStrokes}. As variáveis numéricas ou categóricas ? \footnote{Há apenas dois valores possíveis para cada variável, e nos dois casos eles descrevem categorias, portanto são variáveis categóricas, cada uma.}
\end{exercise}

\subsection{Relacionamento entre as variáveis}
\label{variableRelations}

Muitas análises são motivadas pela procura de relacionamentos entre duas ou mais variáveis. Um cientista social deve responder algumas das questões seguintes:
\begin{enumerate}
\setlength{\itemsep}{0mm}
\item[(1)]\label{fedSpendingPovertyQuestion} O gasto da união, na média, alto ou baixo em municípios com altos índices de pobreza ?
\item[(2)]\label{ownershipMultiUnitQuestion} Se o número de casas próprias é menor que a média nacional por municípios, e a porcentagem de multi-unidade de habitação está acima ou a baixo da média nacional ?
\item[(3)]\label{isAverageIncomeAssociatedWithSmokingBans} Quais municípios tem uma alta média de renda: destes quais tem leis contra o fumo e quais não ?
\end{enumerate}

Para responder estas questões, os dados devem ser coletados, tais como \data{county} conjunto de dados mostrando no Table~\ref{countyDF}. Examinando resumo de estatísticas  \data{summary statistics} poderia fornecer alguma ideia sobre cada uma das três questões sobre municípios. Adicionalmente, gráficos podem ser usados para a visualização dos dados que são sumarizados e também úteis para responder as questões.

\indexthis{Scatterplots}{scatterplot} são um dos tipos de gráficos usados para estudar o relacionamento entre duas variáveis. Figure~\ref{county_fed_spendVsPoverty} compara as variáveis \var{fed\_\hspace{0.3mm}spend} e \var{poverty}. Cada ponto plotado representa um único município. Neste caso, os pontos destacados correspondem a County~1088 in o \data{county} conjunto de dados: Owsley County, Kentucky, o qual tem a pobreza a uma taxa de 41.5\% e um gasto federal de \$21.50 per capita. O scatterplot sugere um relacionamento entre duas variáveis: municípios com alta taxa de pobreza que também tem levemente uma tendência de maior gasto federal. Nós podemos pensar como e porque este relacionamento existe e investigar cada ideia para determinar qual é a explicação mais razoável.

\begin{figure}
\centering
\includegraphics[width=0.8\textwidth]{ch_intro_to_data/figures/county_fed_spendVsPoverty/county_fed_spendVsPoverty}
\caption{Um scatterplot mostrando \var{fed\_\hspace{0.3mm}spend} por \var{poverty}. Owsley County of Kentucky, com uma taxa de pobreza de 41.5\% e o gasto federal \$21.50 per capita, está destacado.}
\label{county_fed_spendVsPoverty}
\end{figure}

\begin{exercise}
Examine as variáveis no \data{email50} conjunto de dados, as quais estão descritas em Table~\vref{email50Variables}. Crie duas questões sobre o relacionamento entre aquelas variáveis do seu interesse.\footnote{Duas questões de amostra: (1) Intuição sugere que se há muitas quebras de linhas em um email então há também a tendência de ter muitos caracteres: Isto é verdade ? (2)~Há uma conexão entre se um formato de email é texto plano (ou HTML) ou se isto é uma mensagem de spam ?}
\end{exercise}

O \var{fed\_\hspace{0.3mm}spend} e \var{poverty} variáveis são ditos associados por que o gráfico mostra um padrão distinto. Quando duas variáveis mostram alguma conexão com outra, são chamadas variáveis \term{associadas}. Variáveis associadas podem também serem chamados variáveis \term{dependentes} e vice-versa.

\begin{example}{Este exemplo examina o relacionamento entre proprietários de casas e a porcentagem de estruturas de unidades múltiplas (ex. apartamentos, condomínios), o qual visualizando usando o scatterplot em Figure~\ref{multiunitsVsOwnership}. Há variáveis associadas ?}
Isto mostra que na maior fração de unidades em estruturas de unidades múltiplas, o baixo taxa de proprietários de moradia própria. Desde que haja um relacionamento entre as variáveis, elas são ditas associadas.
\end{example}

\begin{figure}
   \centering
   \includegraphics[width=0.8\textwidth]{ch_intro_to_data/figures/multiunitsVsOwnership/multiunitsVsOwnership}
   \caption{Um scatterplot dos quem tem casa própria versus o percentual de unidade que estão em estruturas de multiunidades para todos 3,143 municípios. Leitores interessados podem ver na image uma terceira variável adicional, população do município, presente em  \oiRedirect{textbook-homeownership_vs_multi_unit_structures}{www.openintro.org/stat/down/MHP.png}.}
   \label{multiunitsVsOwnership}
\end{figure}

Por que existe uma tenendência descendente em Figure~\ref{multiunitsVsOwnership} -- países com mais unidade em estruturas de multiunidade estão associados a baixa taxa de pessoas com casa própria -- aquelas variáveis são distas \termsub{negativamente associadas}{negativamente associadas}. A \term{associação positiva} é mostrado o relacionamento entre\var{poverty} e \var{fed\_\hspace{0.3mm}spend} variáveis representada na Figure~\ref{county_fed_spendVsPoverty}, onde os municípios com mais altas taxas de pobreza tendem a receber mais ajuda federal per capita.

se duas variáveis são não associadas, então elas podem ser ditas \term{independentes}. Isto por que, duas variáveis são independentes se não evidência de relacionamento entre elas.


\begin{termBox}{\tBoxTitle{Associada ou independente, não ambos}
Um par de variáveis é cada uma mostradada de uma forma (associada) ou não (independente). Não podem ser ambas mostradas como associadas e independentes ao mesmo tempo.}
\end{termBox}

\index{data!county|)}

%%%%%
\section[Princípios da coleta de dados]{Princípios da coleta de dados \sectionvideohref{youtube-2N_bkiyTiXU&list=PLkIselvEzpM6pZ76FD3NoCvvgkj_p-dE8}}
\label{overviewOfDataCollectionPrinciples}

\index{sample|(}
\index{population|(}

O primeiro passo na condução de uma pesquisa é identificar os tópicos ou questões que serão investigadas. Uma clara e definida questão é importante na indentificação sobre quais assunto e casos deverão ser estudados and que variáveis são importantes. Isto também importante considerar \emph{como} os dados são coletados o quanto eles são confiáveis e ajuda a atingir os objetivos da pesquisa.

\subsection{Populações e amostras}
\label{populationsAndSamples}

Considere as três questões de pesquisa que se seguem:

\begin{enumerate}
\setlength{\itemsep}{0mm}
\item Qual é a média de mercúrio contida nos peixes espadas do ocenano Atlântico?
\item\label{timeToGraduationQuestionForUCLAStudents} Nos últmos 5 anos, qual o tempo médio para completar uma graduação para os estudantes na universidade de Duke?
\item\label{identifyPopulationOfStentStudy} Uma nova droga consegue uma redução do número de mortes em pacientes com doênças cardíacas graves ?
\end{enumerate}

Cada questão da pesquisa refere-se a \term{pupulação}. Na primeira questão, a população estudada são todos os peixes-espada do ocenando Atlântico, e cada peixe representa um caso. Algumas vezes, isto é muito caro coletar dados de cada caso na população. Em veze disso, uma amostra é tirada. Uma \term{amostra} representa um subconjunto dos casos e é frequententemente uma pesquena fração da população. Por agora, 60 peixes-espada (ou algum outro número) na pupulação deve ser selecionada, e esta amostra pode ser usada para prover uma estimativa da media da população e responder a questão levantada pela pesquisa.

\begin{exercise} \label{identifyingThePopulationForTwoQuestionsInPopAndSampSubsection}
    Para a segunda e terceira questão, identificar a população alvo e o que representa o caso individual.\footnote{(\ref{timeToGraduationQuestionForUCLAStudents}) Note que a primeira questão é unicamente relevante para estudantes que completem sua graduação; a média não pode ser computada usando um estudante que nunca terminou sua graduação. Portanto, apenas estudantes de Duke que estão cursando são considerados. Cada um destes estudantes pdeveria representar um caso individual. (\ref{identifyPopulationOfStentStudy}) Uma pessoa com doença cardiáca grave representa um caso. A população inclui todas as pessoas com doênças cardiácas grave.}
\end{exercise}

\subsection{Evidência anedótica}
\label{anecdotalEvidenceSubsection}

Consider as seguintes possibilidades de respostas  as questões de pesquisa
\begin{enumerate}

\item Um homem no noticiário foi envenenado com mercúrio por comer peixe-espada contaminado, então a média de concentração de mercúrio nestes peixes está perigosamente alta.
\item\label{iKnowThreeStudentsWhoTookMoreThan7YearsToGraduateAtDuke} Eu conheci dois estudantes que levaram mais de 7 anos para se graduarem em Duke, então a graduação em duque custa mais tempo quem em outras universidades.

\item\label{myFriendsDadDiedAfterSulphinpyrazon} O pai do meu amigo morreu de ataque cardiáco logo após tomar um novo remédio, então este remédio não funciona.
\end{enumerate}
Cada conclusão é basead em dados. De qualquer forma existem dois problemas. Primeiro, os dados unicamente representam um ou dois casos. Segundo, e mais importante, não é claro que estes casos realmente representam a população. Dados coletados a esmo sem o cuidado necessário são chamados \term{evidência anedótica}.

\setlength{\captionwidth}{\textwidth-80mm}
\begin{figure}
\centering
\hspace{8mm}\includegraphics[width=55mm]{ch_intro_to_data/figures/mnWinter/mnWinter}\hspace{4mm}
\begin{minipage}[b]{\textwidth - 80mm}
   \caption[evidência anedótica]{Em Fevereriro 2010 alguns ``especialistas'' na mídia citaram uma grande nevada como evidência válida contra o aquecimento global. E o comediante Jon Stewart comentou ``É uma nevada, em uma região, de um único país.''
   \label{mnWinter}}
\end{minipage}
\end{figure}
\setlength{\captionwidth}{\mycaptionwidth}

\begin{termBox}{\tBoxTitle{Evidência anedótica}
Seja cuidadoso com os dados coletados a esmo. Tais evidências podem ser verdadeiras e verificaveis, mas podem representar apenas casos extraordinários.}
\end{termBox}

Evidência anedótica é tipicamente compasta pela confusão dos casos não comuns, que relembramos com extraordinárias características. Seria mais comun,  nós lembrarmos de duas pessoas que levaram 7 anos para se graduarem que 6 outras que se graduaram em 4 anos. Portanto o correto seria examinar uma amorta com muitos casos que representam a população.

\subsection{Amostrando a população}

\index{sample!random sample|(}
Nós devemos tentar estimar o tempo da graduação em Duke nos últimos 5 anos coletatando amostras dos estudantes. Todos os graduados nos últimos 5 anos representam a \emph{população}\index{população}, e graduados selecionados são chamados \emph{amostra}\index{amostra}. Em geral, nós sempre procuramos selecionar \emph{aleatóriamente} a amostra da população. O mais básico tipo de seleção aleatória é equivalente a como são sorteados. Por exemplo, nos graduandos selecionados, nós poderiamos escrever o nome de cada um num papelsinho misturá-los e tirar 100. Os nomes escolhidos representariam uma amostra de 100 alunos.

\begin{figure}[ht]
\centering
\includegraphics[width=0.47\textwidth]{ch_intro_to_data/figures/popToSample/popToSampleGraduates}
\caption{Neste gráfico, cinco graduandos são aleatóriamente selecionados da população para serem incluído na amostra.}
\label{popToSampleGraduates}
\end{figure}


Por que selecionamos aleatóriamente? Por que não apenas escolher uma amostra qualquer ? Considerando o cenários seguinte.

\begin{example}{Suponha que nós pedimos para uma estudante  que é da nutrição para selecionar vários estudantes da graduação para o estudo. Que tipo de estudante você imaginari que ela coletaria? Você acharia que a coleçao dela seria representativa de todos os estudantes da graduação ?}
Talvêz ela escolha desproporcionamente um número de estudantes da área de saúde, ou talvez sua seleção fosse bem representativa da pupulação. Quando a seleção é feita manualmente, no corremos o risco de termos uma amostra \emph{enviezada}, mesmo se este viéz foi não intencional ou de difícil identificação.
\end{example}

\begin{figure}
\centering
\includegraphics[width=0.47\textwidth]{ch_intro_to_data/figures/popToSample/popToSubSampleGraduates}
\caption{Em vez de amostrar todos os estudante igualemten, uma estuante de nutrição inadivertidamente escolhe graduandos da área de saúde com mais frequência.}
\label{popToSubSampleGraduates}
\end{figure}

Se alguém escolhe estamente quais estudantes serão incluídos na amostra, é perfeitamente possível que esta amostra tenha tendncias aos interesses desta pessoa, mesmo que intencionalmente. Isto introduz um viéz a amostra. Amostrando aleatóriamente ajuda a resolver ester problema, A amostragem mais básica é chamda \term{amostra aleatória simples}, e é equivalente usar um recipiente com os casos todos misturados para a seleção. Isto significa  cada caso tem igual chance de ser incluído ou não implicando conexão entre os casos e a amostra.

\APVersion{Sometimes a simple random sample is difficult to implement and an alternative method is helpful. One such substitute is a \term{systematic sample}, where one case is sampled after letting a fixed number of others, say 10 other cases, pass by. Since this approach uses a mechanism that is not easily subject to personal biases, it often yields a reasonably representative sample. This book will focus on random samples since the use of systematic samples is uncommon and requires additional considerations of the context.}{}

The act of taking a simple random sample helps minimize bias, however, bias can crop up in other ways.
Even when people are picked at random, e.g. for surveys, caution must be exercised if the \term{non-response} \index{sample!non-response|textbf} is high. For instance, if only 30\% of the people randomly sampled for a survey actually respond, then it is unclear whether the results are \term{representative} of the entire population. This \term{non-response bias} \index{sample!non-response bias|textbf} can skew results.

\begin{figure}[h]
\centering
\includegraphics[width=0.5\textwidth]{ch_intro_to_data/figures/popToSample/surveySample}
\caption{Due to the possibility of non-response, surveys studies may only reach a certain group within the population. It is difficult, and often times impossible, to completely fix this problem.}
\label{surveySample}
\end{figure}

Another common downfall is a \term{convenience sample}\index{sample!convenience sample}, where individuals who are easily accessible are more likely to be included in the sample. For instance, if a political survey is done by stopping people walking in the Bronx, this will not represent all of New York City. It is often difficult to discern what sub-population a convenience sample represents.

\begin{exercise}
We can easily access ratings for products, sellers, and companies through websites. These ratings are based only on those people who go out of their way to provide a rating. If 50\% of online reviews for a product are negative, do you think this means that 50\% of buyers are dissatisfied with the product?\footnote{Answers will vary. From our own anecdotal experiences, we believe people tend to rant more about products that fell below expectations than rave about those that perform as expected. For this reason, we suspect there is a negative bias in product ratings on sites like Amazon. However, since our experiences may not be representative, we also keep an open mind.}
\end{exercise}

\index{sample!random sample|)}
\index{population|)}
\index{sample|)}

\subsection{Explanatory and response variables}
\label{explanatoryAndResponse}

\index{data!county|(}

Consider the following question from page~\pageref{fedSpendingPovertyQuestion} for the \data{county} data set:
\begin{enumerate}
\item[(1)]
	Is federal spending, on average, higher or lower in counties with high rates of poverty?
\end{enumerate}
If we suspect poverty might affect spending in a county, then poverty is the \term{explanatory} variable and federal spending is the \term{response} variable in the relationship.\footnote{Sometimes the explanatory variable is called the \term{independent} variable and the response variable is called the \term{dependent} variable. However, this becomes confusing since a \emph{pair} of variables might be independent or dependent, so we avoid this language.} If there are many variables, it may be possible to consider a number of them as explanatory variables.

\begin{tipBox}{\tipBoxTitle{Explanatory and response variables}
To identify the explanatory variable in a pair of variables, identify which of the two is suspected of affecting the other and plan an appropriate analysis.

\hspace{10mm}\includegraphics[height=0.34in]{ch_intro_to_data/figures/expResp/expResp}}
\end{tipBox}

\begin{caution}{association does not imply causation}{Labeling variables as \emph{explanatory} and \emph{response} does not guarantee the relationship between the two is actually causal, even if there is an association identified between the two variables. We use these labels only to keep track of which variable we suspect affects the other.}
\end{caution}

In some cases, there is no explanatory or response variable. Consider the following question from page~\pageref{ownershipMultiUnitQuestion}:
\begin{enumerate}
\item[(2)]
	If homeownership is lower than the national average in one county, will the percent of multi-unit structures in that county likely be above or below the national average?
\end{enumerate}
It is difficult to decide which of these variables should be considered the explanatory and response variable, i.e. the direction is ambiguous, so no explanatory or response labels are suggested here.

\index{data!county|)}

\subsection{Introducing observational studies and experiments}

There are two primary types of data collection: observational studies and experiments.

Researchers perform an \term{observational study} when they collect data in a way that does not directly interfere with how the data arise. For instance, researchers may collect information via surveys, review medical or company records, or follow a \term{cohort} of many similar individuals to study why certain diseases might develop. In each of these situations, researchers merely observe the data that arise. In general, observational studies can provide evidence of a naturally occurring association between variables, but they cannot by themselves show a causal connection.

When researchers want to investigate the possibility of a causal connection, they conduct an \term{experiment}. Usually there will be both an explanatory and a response variable. For instance, we may suspect administering a drug will reduce mortality in heart attack patients over the following year. To check if there really is a causal connection between the explanatory variable and the response, researchers will collect a sample of individuals and split them into groups. The individuals in each group are \emph{assigned} a treatment. When individuals are randomly assigned to a group, the experiment is called a \term{randomized experiment}. For example, each heart attack patient in the drug trial could be randomly assigned,  perhaps by flipping a coin, into one of two groups: the first group receives a \term{placebo} (fake treatment) and the second group receives the drug. See the case study in Section~\ref{basicExampleOfStentsAndStrokes} for another example of an experiment, though that study did not employ a placebo.

\begin{tipBox}{\tipBoxTitle{association $\neq$ causation}
In general, association does not imply causation, and causation can only be inferred from a randomized experiment.}
\end{tipBox}


%%%%%
\section[Observational studies and sampling strategies]{Observational studies and sampling strategies \sectionvideohref{youtube-KyuaX10l3GQ&list=PLkIselvEzpM6pZ76FD3NoCvvgkj_p-dE8}}

\subsection{Observational studies}

Generally, data in observational studies are collected only by monitoring what occurs, while experiments require the primary explanatory variable in a study be assigned for each subject by the researchers.

Making causal conclusions based on experiments is often reasonable. However, making the same causal conclusions based on observational data can be treacherous and is not recommended. Thus, observational studies are generally only sufficient to show associations.

\begin{exercise} \label{sunscreenLurkingExample}
Suppose an observational study tracked sunscreen use and skin cancer, and it was found that the more sunscreen someone used, the more likely the person was to have skin cancer. Does this mean sunscreen \emph{causes} skin cancer?\footnote{No. See the paragraph following the exercise for an explanation.}
\end{exercise}

Some previous research tells us that using sunscreen actually reduces skin cancer risk, so maybe there is another variable that can explain this hypothetical association between sunscreen usage and skin cancer. One important piece of information that is absent is sun exposure. If someone is out in the sun all day, she is more likely to use sunscreen \emph{and} more likely to get skin cancer. Exposure to the sun is unaccounted for in the simple investigation.
\begin{center}
\includegraphics[height=1.0in]{ch_intro_to_data/figures/variables/sunCausesCancer}
\end{center}
% Some studies:
% http://www.sciencedirect.com/science/article/pii/S0140673698121682
% http://archderm.ama-assn.org/cgi/content/abstract/122/5/537
% Study with a similar scenario to that described here:
% http://onlinelibrary.wiley.com/doi/10.1002/ijc.22745/full

Sun exposure is what is called a \term{confounding variable},\footnote{Also called a \term{lurking variable}, \term{confounding factor}, or a \term{confounder}.} which is a variable that is correlated with both the explanatory and response variables. While one method to justify making causal conclusions from observational studies is to exhaust the search for confounding variables, there is no guarantee that all confounding variables can be examined or measured.

In the same way, the \data{county} data set is an observational study with confounding variables, and its data cannot easily be used to make causal conclusions.

\begin{exercise}
Figure~\ref{multiunitsVsOwnership} shows a negative association between the homeownership rate and the percentage of multi-unit structures in a county. However, it is unreasonable to conclude that there is a causal relationship between the two variables. Suggest one or more other variables that might explain the relationship visible in Figure~\ref{multiunitsVsOwnership}.\footnote{Answers will vary. Population density may be important. If a county is very dense, then this may require a larger fraction of residents to live in multi-unit structures. Additionally, the high density may contribute to increases in property value, making homeownership infeasible for many residents.}
\end{exercise}

Observational studies come in two forms: prospective and retrospective studies. A \term{prospective study} identifies individuals and collects information as events unfold. For instance, medical researchers may identify and follow a group of similar individuals over many years to assess the possible influences of behavior on cancer risk. One example of such a study is The Nurses' Health Study, started in 1976 and expanded in 1989.\footnote{\oiRedirect{textbook-channing_nurse_study}{www.channing.harvard.edu/nhs}} This prospective study recruits registered nurses and then collects data from them using questionnaires. \termsub{Retrospective studies}{retrospective studies} collect data after events have taken place, e.g. researchers may review past events in medical records. Some data sets, such as \data{county}, may contain both prospectively- and retrospectively-collected variables. Local governments prospectively collect some variables as events unfolded (e.g. retails sales) while the federal government retrospectively collected others during the 2010 census (e.g. county population counts).

\subsection{Four sampling methods (special topic)}
\label{fourSamplingMethods}
\label{threeSamplingMethods}

Almost all statistical methods are based on the notion of implied randomness. If observational data are not collected in a random framework from a population, these statistical methods -- the estimates and errors associated with the estimates -- are not reliable. Here we consider four random sampling techniques: simple, stratified, cluster, and multistage sampling. Figures~\ref{simple_stratified} and~\ref{cluster_multistage} provide graphical representations of these techniques.

\begin{figure}
\centering
\includegraphics[width=\textwidth]{ch_intro_to_data/figures/samplingMethodsFigure/simple_stratified}
\caption{Examples of simple random\index{sample!simple random sampling} and stratified sampling\index{sample!stratified sampling}. In the top panel, simple random sampling was used to randomly select the 18 cases. In the bottom panel, stratified sampling was used: cases were grouped into strata, then simple random sampling was employed within \mbox{each stratum}.}
\label{simple_stratified}
\end{figure}

\termsub{Simple random sampling}{sample!simple random sampling} is probably the most intuitive form of random sampling. Consider the salaries of Major League Baseball (MLB) players, where each player is a member of one of the league's 30 teams. To take a simple random sample of 120 baseball players and their salaries from the 2010 season, we could write the names of that season's 828 players onto slips of paper, drop the slips into a bucket, shake the bucket around until we are sure the names are all mixed up, then draw out slips until we have the sample of 120 players. In general, a sample is referred to as ``simple random'' if each case in the population has an equal chance of being included in the final sample \emph{and} knowing that a case is included in a sample does not provide useful information about which other cases are included.

\termsub{Stratified sampling}{sample!stratified sampling} is a divide-and-conquer sampling strategy. The population is divided into groups called \term{strata}\index{sample!strata|textbf}. The strata are chosen so that similar cases are grouped together, then a second sampling method, usually simple random sampling, is employed within each stratum. In the baseball salary example, the teams could represent the strata, since some teams have a lot more money (up to 4~times as much!). Then we might randomly sample 4 players from each team for a total of 120 players.

Stratified sampling is especially useful when the cases in each stratum are very similar with respect to the outcome of interest. The downside is that analyzing data from a stratified sample is a more complex task than analyzing data from a simple random sample. The analysis methods introduced in this book would need to be extended to analyze data collected using stratified sampling.

\begin{example}{Why would it be good for cases within each stratum to be very similar?}
We might get a more stable estimate for the subpopulation in a stratum if the cases are very similar. These improved estimates for each subpopulation will help us build a reliable estimate for the full population.
\end{example}

In a \termsub{cluster sample}{sample!cluster sample}, we break up the population into many groups, called \termsub{clusters}{sample!cluster}. Then we sample a fixed number of clusters and include all observations from each of those clusters in the sample. A \termsub{multistage sample}{sample!multistage sample} is like a cluster sample, but rather than keeping all observations in each cluster, we collect a random sample within each selected cluster. %Multistage sampling is similar to stratified sampling in its process, except that stratified sampling requires observations be sampled from \emph{every} stratum.

\begin{figure}
\centering
\includegraphics[width=\textwidth]{ch_intro_to_data/figures/samplingMethodsFigure/cluster_multistage}
\caption{Examples of cluster\index{sample!cluster sampling} and multistage sampling\index{sample!multistage sampling}. In the top panel, cluster sampling was used. Here, data were binned into nine clusters, three of these clusters were sampled, and all observations within these three cluster were included in the sample. In the bottom panel, multistage sampling was used.
It differs from cluster sampling in that of the clusters selected, we randomly select a subset of each cluster to be included in the sample.}
\label{cluster_multistage}
\end{figure}

Sometimes cluster or multistage sampling can be more economical than the alternative sampling techniques. Also, unlike stratified sampling, these approaches are most helpful when there is a lot of case-to-case variability within a cluster but the clusters themselves don't look very different from one another. For example, if neighborhoods represented clusters, then cluster or multistage sampling work best when the neighborhoods are very diverse. A downside of these methods is that more advanced analysis techniques are typically required, though the methods in this book can be extended to handle such data.

\begin{example}{Suppose we are interested in estimating the malaria rate in a densely tropical portion of rural Indonesia. We learn that there are 30 villages in that part of the Indonesian jungle, each more or less similar to the next. Our goal is to test 150 individuals for malaria. What sampling method should be employed?}
A simple random sample would likely draw individuals from all 30 villages, which could make data collection extremely expensive. Stratified sampling would be a challenge since it is unclear how we would build strata of similar individuals. However, cluster sampling or multistage sampling seem like very good ideas. If we decided to use multistage sampling, we might randomly select half of the villages, then randomly select 10 people from each. This would probably reduce our data collection costs substantially in comparison to a simple random sample, and this approach would still give us reliable information.
\end{example}


%%%%%

% ex:filetype=tex
\section[Experimentos]{Experimentos \sectionvideohref{youtube-g7JGe_ykB3I&list=PLkIselvEzpM6pZ76FD3NoCvvgkj_p-dE8}}
\label{experimentsSection}

Estudos em que os pesquisadores atribuem tratamentos a casos chamam-se \termsub{experimentos}{experimento}.
Quando esta atribuição inclui randomização, por exemplo tirando cara ou coroa para decidir a qual tratamento um paciente recebe, ela chama-se \term{experimento randomizado}.
Experimentos randomizados são importantíssimos para provar uma ligação causal entre duas variáveis.

\subsection{Principles of experimental design}
\label{experimentalDesignPrinciples}

Experimentos randomizados baseiam-se geralmente em quatro princípios.
\begin{description}
 \item[Controle.]
  Pesquisadores atribuem tratamentos a casos, e eles tentam o melhor de \term{controlar} todas as outras diferenças nos grupos.
  Por exemplo, quando pacientes tomam uma droga em forma de uma pílula, uns pacientes tomam a pílula com apenas um gole de água, enquanto outros com um copo inteiro de água.
  Para controlar o efeito de consumo de água, o doutor pode pedir a todos os pacientes de beber um copo de água de 12 onça com a pílula.
 \item[Randomização.]
  Pesquisadores randomizam pacientes em grupos de tratamento para tomar conta de variáveis que não podem ser controladas.
  Por exemplo, uns pacientes podem ser mais suscetíveis a uma doença que outros devido a sua rotina dietária.
  Randomizando pacientes em grupos de tratamento ajuda nivelar tais diferenças, e também previne bias acidental de entrar no estudo.
 \item[Replicação.]
  Quanto mais casos pesquisadores observam, quanto mais exatamente podem estimar o efeito da variável explanatória na resposta.
  Em um único estudo, \term{replicamos} coletando uma amostra suficientemente grande.
  Adicionalmente, um grupo de científicos pode replicar um estudo inteiro para verificar um resultado anterior.

\begin{figure}
\centering
\includegraphics[width=0.78\textwidth]{ch_intro_to_data/figures/figureShowingBlocking/figureShowingBlocking}
\caption{Agrupamento em blocos} usando uma variável descrevendo o risco de paciente.
 Pacientes são primeiramente divididos entre blocos de baixo e alto risco, em seguida, usando randomização, cada bloco é uniformemente separado nos grupos de tratamento.
 Esta estratégia assegura uma representação uniforme de pacientes em cada grupo de tratamento das duas categorias, de alto e de baixo risco.
\label{figureShowingBlocking}
\end{figure}

\item[Agrupamento em blocos.]
 Pesquisadores de vez em quando sabem ou supõem que variáveis, a não ser o tratamento, influenciam a resposta.
 Nestas circunstâncias, podem primeiramente agrupar indivíduos, baseado nesta variável, em \term{blocos} e depois randomizar casos em cada bloco dos grupos de tratamento.
 Esta estratégia referencia-se com frequência como \term{blocking} (agrupamento em blocos).
 Por exemplo, se estamos olhando o efeito de uma droga nos infartos, poderíamos primeiro separar pacientes no estudo em pacientes de alto e de baixo risco, em seguida aleatoriamente lotear uma metade dos pacientes de cada grupo no grupo de controle e a outra metade no grupo de tratamento, como mostrado em Figura~\ref{figureShowingBlocking}.
 Esta estratégia garante que cada grupo de tratamento tem um número igual de pacientes de alto e baixo risco.
\end{description}

 É importante incorporar os primeiro três princípios de projeto em qualquer estudo, e este livro descreve métodos aplicáveis para analisar dados de tais experimentos.
 Blocking é uma técnica levemente mais avançada, e métodos estatísticos neste livro podem ser estendidos para analisar dados coletados usando blocking.

\subsection{Reduzindo bias em experimentos humanos}
\label{biasInHumanExperiments}

Experimentos randomizados são o alto padrão para coletas de dados, contudo não garantem uma perspectiva isenta para as relações entre causa e efeito em todos os casos.
Estudos humanos servem com exemplo perfeito em que bias pode surgir involuntariamente.
Aqui consideramos um estudo em que uma nova droga foi usada para tratar pacientes de infarto.
\footnote{Anturane Reinfarction Trial Research Group. 1980. Sulfinpyrazone in the prevention of sudden death after myocardial infarction.
New England Journal of Medicine 302(5):250-256.}
Em particular, pesquisadores quiseram saber se a droga diminuiu a mortalidade entre os pacientes.

Estes pesquisadores projetaram um experimento randomizado porque quiseram tirar conclusões sobre o efeito da droga.
Voluntários foram aleatoriamente loteados em dois grupos.
\footnote{Seres humanos são frequentemente referidos como \term{patientes}, \term{voluntários}, ou \term{participantes}.}
Um grupo, o \term{grupo de tratamento}, recebeu a droga.
O outro grupo, chamado o \term{grupo de controle}, recebeu nenhuma droga.

Imagine-te em lugar de uma dessas pessoas no estudo.
Por um lado, se estás no grupo de tratamento, recebes uma nova droga sofisticada que antecipas a ajudar-lhe.
Por outro, uma pessoa no outro grupo não recebe a droga e cruza os braços, esperando que a sua participação não aumente a sua mortalidade.
Estas perspectivas sugerem que há com efeito dois efeitos:
O do nosso interesse é a efetividade da droga, o segundo é o efeito emocional dificilmente quantificado.

Pesquisadores usualmente não são interessados no efeito emocional, que pode predispor o estudo.
Para contornar este problema, pesquisadores não querem os pacientes saberem em qual grupo estão.
Se pesquisadores mantêm os pacientes desinformados sobre seu tratamento, o estudo é \term{cego}.
Mas há um problema:
Se um paciente não recebe o tratamento, ela saberá que está no grupo de controle.
A solução deste problema é dar um tratamento fingido aos pacientes no grupo de controle.
Um tratamento fingido chama-se \term{placebo}, e um placebo efetivo é a chave para tornar um estudo verdadeiramente cego.
Um exemplo clássico de um placebo é uma pílula de açúcar feito para aparecer como a verdadeira pílula de tratamento.
Muitas vezes, um placebo resulta em um pequeno mas verdadeiro melhoramento dos pacientes.
Este efeito chama-se o \term{efeito~de~placebo}

Os pacientes  não são os únicos que podem ser cegados:
doutores e pesquisadores podem involuntariamente influenciar um estudo:
Quando um doutor sabe que um paciente recebeu o verdadeiro tratamento, ela pode inadvertidamente prestar mais atenção ou tratamento a esse paciente do que a um paciente de que ela sabe que recebeu o placebo.
Para guardar contra este bias, que comprovadamente tem um efeito mensurável em uns casos, a maioria dos estudos modernos usam um esquema \term{duplo-cego} em que os doutores ou pesquisadores que interagem com pacientes são desinformados sobre quem recebeu ou não recebeu o tratamento, exatamente como os pacientes.
\footnote{Sempre existem pesquisadores envolvidos que sabem quais pacientes receberam qual tratamento.
  Porém, não interagem com os pacientes do estudo, e não informam os profissionais cegados sobre que está recebendo qual tratamento.}

\begin{exercise}
  Revisite o estudo em Seção~\ref{basicExampleOfStentsAndStrokes} em que pesquisadores verificaram se stents reduziram efetivamente infartos em pacientes de alto risco.
  Isto é um experimento?
  O estudo foi cego?
  Foi duplo-cego?
  \footnote{Os pesquisadores lotaram os pacientes em seus grupos de tratamento, fazendo com que este estudo foi um experimento.
    Porém, os pacientes podiam distinguir qual tratamento receberam, de forma que o estudo não era cego.
    Logo, o estudo não podia ser duplo-cego, já que não era cego.}
\end{exercise}

%%%%%
\section[Examinando dados numéricos]{Examinando dados numéricos \sectionvideohref{youtube-Xm0PPtci3JE&list=PLkIselvEzpM6pZ76FD3NoCvvgkj_p-dE8}}
\label{numericalData}

Nesta seção introduzimos as técnicas para explorar e sumarizar variáveis numéricas.
Os conjuntos de dados \data{email50} e \data{county} da Seção~\ref{dataBasics} fornecem
The \data{email50} and \data{county} data sets from Section~\ref{dataBasics} proporcionam ricas oportunidades para exemplos.
Lembramos que resultados de variáveis numéricas são números sobre os quais é razoável fazer operações aritméticas básicas.

Por exemplo, a variável \var{pop2010}, que representa a população dos counties em 2010, é numérica porque podemos ajuizadamente discutir as diferenças ou proporção das populações em dois países.
De outro lado, códigos de areá ou códigos postais não são numéricas, mas mais propriamente variáveis~categóricas.

\subsection{Scatterplots para dados emparelhados}
\label{scatterPlots}

\index{data!email50|(}

Um \term{scatterplot} proporciona uma vista em cada caso específico dos dados para dois variáveis numéricas.
Em figura~\vref{county_fed_spendVsPoverty}, um scatterplot foi usado para examinar como despesas federais e pobreza se relacionam no conjunto de dados \data{county}.
Outro scatterplot vê-se em figura~\ref{email50LinesCharacters}, comparando o número de quebras de linha (\var{line\_\hspace{0.3mm}breaks}) e o número de caráteres (\var{num\_\hspace{0.3mm}char}) em e-mails do conjunto de dados \data{email50}.
Em qualquer scatterplot, cada ponto representa um caso especifico.
Porque há 50 casos em \data{email50}, há 50 pontos em figura~\ref{email50LinesCharacters}.

\textC{\setlength{\captionwidth}{0.9\textwidth}}

\begin{figure}[h]
   \centering
   \includegraphics[width=0.8\textwidth]{ch_intro_to_data/figures/email50LinesCharacters/email50LinesCharacters}
   \caption{Um scatterplot de \var{line\_\hspace{0.3mm}breaks} versos \var{num\_\hspace{0.3mm}char} para os dados \data{email50}.}
   \label{email50LinesCharacters}
\end{figure}

\textC{\setlength{\captionwidth}{\mycaptionwidth}}

Para pôr o número de caráteres em perspectiva, este parágrafo tem 999 caráteres.
Olhando à figura~\ref{email50LinesCharacters}, parece que uns e-mails foram incrivelmente verbosos!
Investigando mais, pudermos descobrir que a maioria dos e-mails longos usam o formato HTML, o que significa que a maioria dos carateres nesses e-mails é usada para formatar o e-mail, ao invés de fornecer texto.

\begin{exercise}
 O quê revelam scatterplots sobre os dados, e como poderiam ser uteis?
 \footnote{
  Respostas podem variar.
  Scatterplots são uteis para rapidamente reparar associações entre variáveis relacionadas, quer que estas associações vêm na forma de tendências simples ou esses relacionamentos são mais complexos.
 }
\end{exercise}

\index{data!cars|(}
\begin{example}{
  Considera o novo conjunto de dados de 54 caráteres de duas variáveis:
  preço de veiculo e~peso.
  \footnote{
   Conjunto de dados de \oiRedirect{textbook-1993_car_data}{www.amstat.org/publications/jse/v1n1/datasets.lock.html}
  }
  Um scatterplot de preço de veiculo versus peso é mostrado em figura~\ref{carsPriceVsWeight}.
  O quê pode ser dito sobre a relação entre estas duas variáveis?
}
A relação é evidentemente não-linear, como destacado pela linha tracejada.
Isto difere de scatterplots antigos que vimos, tal como figura~\vref{county_fed_spendVsPoverty} e figura~\ref{email50LinesCharacters}, que mostram relações bem lineares.

\begin{figure}[h]
   \centering
   \includegraphics[width=0.8\textwidth]{ch_intro_to_data/figures/carsPriceVsWeight/carsPriceVsWeight}
   \caption{Um scatterplot de \var{preço} versus \var{peso} para 54 carros.}
   \label{carsPriceVsWeight}
\end{figure}
\end{example}
\index{data!cars|)}

\begin{exercise}
 Descreva duas variaveis que teriam uma associação em forma de um ferradura em um scatterplot

 \footnote{
   Considera o caso em que o eixo vertical representa algo ``bom'' e o eixo horizontal representa algo que é só bom com moderação.
  Saúde e consumo de água cabem esta descrição porque água torna-se toxica quando consumido em quantidades excessivas.
 }
\end{exercise}

\subsection{Traçados de pontos e a média}
\label{dotPlot}

As vezes duas variáveis são uma demais:
Só uma pode ser interessante.
Nestes casos, um traçado de pontos fornece a exposição mais básica entre todas.
Um~\term{traçado de pontos} é um scatterplot de uma variável;
um exemplo usando o número de carateres de 50 e-mails é mostrado em figura~\ref{emailCharactersDotPlot}.
Uma versão empilhada deste traçado de pontos é mostrado em figura~\ref{emailCharactersDotPlotStacked}.

\begin{figure}[h]
   \centering
   \includegraphics[width=\textwidth]{ch_intro_to_data/figures/emailCharactersDotPlot/emailCharactersDotPlot}
   \caption{Um traçado de pontos de \var{num\_\hspace{0.3mm}char} para o conjunto de dados \data{email50}.}
   \label{emailCharactersDotPlot}
\end{figure}

\begin{figure}[h]
   \centering
   \includegraphics[width=0.72\textwidth]{ch_intro_to_data/figures/emailCharactersDotPlot/emailCharactersDotPlotStacked}
   \caption{
    Um traçado de pontos empilhado de \var{num\_\hspace{0.3mm}char} para o conjunto de dados \data{email50}.
    Os~valora foram arredondados para o mais próximo 2,000 neste traçado.
   }
   \label{emailCharactersDotPlotStacked}
\end{figure}

O \term{meio}, as vezes chamado a \indexthis{média}{mean!average}, é uma maneira comum para medir o centro de uma distribuição de dados.
Para encontrar o número médio de caráteres nos 50 e-mails, adicionamos todas as contagens de caráteres e dividimos pelo número de e-mails.
Para facilitar o cálculo, o número de caráteres é alistado em múltiplos de mil e arredondado ao primeiro~decimal.
\begin{eqnarray}
\bar{x} = \frac{21.7 + 7.0 + \cdots + 15.8}{50} = 11.6
\label{sampleMeanEquation}
\end{eqnarray}
O meio de uma amostra é frequentemente etiquetado $\bar{x}$\marginpar[\raggedright$\bar{x}$\\\footnotesize meio de\\ amostra]{\raggedright$\bar{x}$\\\footnotesize meio de\\ amostra}.
A letra $x$ é usada como símbolo reservado para a variável de interesse, \var{num\_\hspace{0.3mm}char}, e o bar sobre a letra $x$ simboliza que o número médio de caráteres nos 50 e-mails foi 11,600.
É proveitoso pensar do meio como ponto de balança da distribuição.
O meio de amostra é indicado como triangulo em figurar~\ref{emailCharactersDotPlot} e~\ref{emailCharactersDotPlotStacked}.

\begin{termBox}{\tBoxTitle{Mean}%
  O meio de amostra de uma variável numérica é calculado como soma de todas as observações dividida pelo número de observações:
\begin{eqnarray}
\bar{x} = \frac{x_1+x_2+\cdots+x_n}{n}
\label{meanEquation}
\end{eqnarray}
onde $x_1, x_2, \dots, x_n$ representam os $n$ valores observados.}
\end{termBox}\marginpar[\raggedright\vspace{-8mm}

$n$\\\footnotesize tamanho de amostra]{\raggedright\vspace{-8mm}

$n$\\\footnotesize tamanho de amostra}\vspace{-2mm}

\begin{exercise}
 Examine Equações~\eqref{sampleMeanEquation} e~\eqref{meanEquation} acima.
 Ao quê corresponde $x_1$?
 E $x_2$?
 Podes inferir um significado geral ao que $x_i$ representa?
 \footnote{
  $x_1$ corresponde ao número de caráteres no primerio e-mail na amostra (21.7, em milhares), $x_2$ ao número de caráteres no segundo e-mail (7.0, em milhares), e $x_i$ corresponde ao número de caráteres no e-mail número $i$ no conjunto de dados.
 }
\end{exercise}

\begin{exercise}
 O quê foi $n$ nesta amostra de e-mails?
 \footnote{O tamanho da amostra foi $n=50$.}
\end{exercise}

O conjunto de dados \data{email50} representa uma amostra de uma população maior de e-mails que foi recebida em janeiro e março.
Poderíamos calcular o meio desta população do mesmo jeito como no meio de amostra, porém, o meio de população tem um letreiro especial:
$\mu$\marginpar[\raggedright$\mu$\\\footnotesize meio de \\ população]{\raggedright$\mu$\\\footnotesize meio de\\ população}.
\index{Greek!mu@mu ($\mu$)} O símbolo $\mu$ é a letra grega \emph{mu} e representa a média de todas as observações na população.
De vez em quando um subscrito, como $_x$, é usado para representar a qual variável o meio de população refere, por exemplo $\mu_x$.

\begin{example}{
  O número médio de todos os caráteres em todos os e-mails pode ser estimado usando os dados de amostra.
  Baseando-se na amostra de 50 e-mails, o que seria uma estimativa razoável de $\mu_x$, o número médio de caráteres em todos os e-mail no conjunto de dados \data{email}?
 (Lembre-se que \data{email50} é uma amostra de \data{email}.)}
 O meio de amostra, 11,600, pode proporcionar uma estimativa razoável de $\mu_x$.
 Enquanto este número não será perfeito, fornece uma estimativa de ponto do meio da população.
 Em capítulo~\ref{foundationsForInference} e além, desenvolvermos ferramentas para caraterizar a exatidão de estimativas de ponto, e encontraremos que estimativas de pontos que são baseadas em amostras maiores tendem a ser mais exatas que as baseadas em amostras menores.
\end{example}

\begin{example}{

  Poderiamos gostar de calcular a renda média por pessoa nos estados unidos.
  Para isto, poderiamos primeiro pensar em medir a média das rendas por pessoa entre todos os 3,143 counties no conjunto de dados \data{county}.
  O que seria uma abordagem melhor?
} \label{wtdMeanOfIncome}
O conjunto de dados \data{county} e especial em que cada county de fato representa muitas pessoas individuais.
Se calculássemos simplesmente a média da variável \var{renda}, trataríamos conties com 5,000 e 5,000,000 habitantes de forma uniforma nos cálculos.
Em vez disto, deveriamos calcular a renda total de cada county, e somar as somar de todos os counties, e depois dividir pela quantidade de pessoas em todos os counties.
Se calculassemos a média simples da renda por pessoa entre todos os counties, o resultado teria sido só \$22,504.70!
\end{example}

Exe
Exemplo~\ref{wtdMeanOfIncome} usou o que se chama um \term{meio ponderado}\index{mean!weighted mean}, o qual não será um assunto chave neste livro.
Porém, preparamos um suplemento online sobre meios ponderados para os leitores interessados:
\begin{center}
\oiRedirect{textbook-weighted_mean_supplement}{www.openintro.org/stat/down/supp/wtdmean.pdf}
\end{center}

\subsection{Histograms and shape}
\label{histogramsAndShape}

Dot plots show the exact value for each observation. This is useful for small data sets, but they can become hard to read with larger samples. Rather than showing the value of each observation, we prefer to think of the value as belonging to a \emph{bin}. For example, in the \data{email50} data set, we create a table of counts for the number of cases with character counts between 0 and 5,000, then the number of cases between 5,000 and 10,000, and so on. Observations that fall on the boundary of a bin (e.g. 5,000) are allocated to the lower bin. This tabulation is shown in Table~\ref{binnedNumCharTable}. These binned counts are plotted as bars in Figure~\ref{email50NumCharHist} into what is called a \term{histogram}, which resembles the stacked dot plot shown in Figure~\ref{emailCharactersDotPlotStacked}.

\begin{table}[ht]
\centering\small
\begin{tabular}{l ccc ccc ccc c}
  \hline
Characters & \\
(in thousands) & \raisebox{1.5ex}[0pt]{0-5} & \raisebox{1.5ex}[0pt]{5-10} & \raisebox{1.5ex}[0pt]{10-15} & \raisebox{1.5ex}[0pt]{15-20} & \raisebox{1.5ex}[0pt]{20-25} & \raisebox{1.5ex}[0pt]{25-30} & \raisebox{1.5ex}[0pt]{$\cdots$} & \raisebox{1.5ex}[0pt]{55-60} & \raisebox{1.5ex}[0pt]{60-65} \\
  \hline
Count & 19 & 12 & 6 & 2 & 3 & 5 & $\cdots$ & 0 & 1 \\
  \hline
\end{tabular}
\caption{The counts for the binned \var{num\_\hspace{0.3mm}char} data.}
\label{binnedNumCharTable}
\end{table}

\begin{figure}[bth]
   \centering
   \includegraphics[width=0.82\textwidth]{ch_intro_to_data/figures/email50NumCharHist/email50NumCharHist}
   \caption{A histogram of \var{num\_\hspace{0.3mm}char}. This distribution is very strongly skewed to the right.\index{skew!example: very strong}}
   \label{email50NumCharHist}
\end{figure}

Histograms provide a view of the \term{data density}. Higher bars represent where the data are relatively more common. For instance, there are many more emails with fewer than 20,000 characters than emails with at least 20,000 in the data set. The bars make it easy to see how the density of the data changes relative to the number of characters.

Histograms are especially convenient for describing the shape of the data distribution\label{shapeFirstDiscussed}. Figure~\ref{email50NumCharHist} shows that most emails have a relatively small number of characters, while fewer emails have a very large number of characters. When data trail off to the right in this way and have a longer right \hiddenterm{tail}\index{skew!tail}, the shape is said to be \termsub{right skewed}{skew!right skewed}.\footnote{Other ways to describe data that are skewed to the right: \termni{skewed to the right}, \termni{skewed to the high end}, or \termni{skewed to the positive end}.}

Data sets with the reverse characteristic -- a long, thin tail to the left -- are said to be \termsub{left skewed}{skew!left skewed}. We also say that such a distribution has a long left tail. Data sets that show roughly equal trailing off in both directions are called \term{symmetric}.\index{skew!symmetric}

\begin{termBox}{\tBoxTitle{Long tails to identify skew}%
When data trail off in one direction, the distribution has a \term{long tail}. \index{skew!long tail|textbf} If a distribution has a long left tail, it is left skewed. If a distribution has a long right tail, it is right skewed.}
\end{termBox}

\begin{exercise}
Take a look at the dot plots in Figures~\ref{emailCharactersDotPlot} and~\ref{emailCharactersDotPlotStacked}. Can you see the skew in the data? Is it easier to see the skew in this histogram or the dot plots?\footnote{The skew is visible in all three plots, though the flat dot plot is the least useful. The stacked dot plot and histogram are helpful visualizations for identifying skew.}
\end{exercise}

\begin{exercise}
Besides the mean (since it was labeled), what can you see in the dot plots that you cannot see in the histogram?\footnote{Character counts for individual emails.}
\end{exercise}

In addition to looking at whether a distribution is skewed or symmetric, histograms can be used to identify modes. A \term{mode} is represented by a prominent peak in the distribution.\footnote{Another definition of mode, which is not typically used in statistics, is the value with the most occurrences. It is common to have \emph{no} observations with the same value in a data set, which makes this other definition useless for many real data sets.} There is only one prominent peak in the histogram of \var{num\_\hspace{0.3mm}char}.

Figure~\ref{singleBiMultiModalPlots} shows histograms that have one, two, or three prominent peaks. Such distributions are called \termsub{unimodal}{modality!unimodal}, \termsub{bimodal}{modality!bimodal}, and \termsub{multimodal}{modality!multimodal}, respectively. Any distribution with more than 2 prominent peaks is called multimodal. Notice that there was one prominent peak in the unimodal distribution with a second less prominent peak that was not counted since it only differs from its neighboring bins by a few observations.

\begin{figure}[h]
   \centering
   \includegraphics[width=\textwidth]{ch_intro_to_data/figures/singleBiMultiModalPlots/singleBiMultiModalPlots}
   \caption{Counting only prominent peaks, the distributions are (left to right) unimodal, bimodal, and multimodal.}
   \label{singleBiMultiModalPlots}
\end{figure}

\begin{exercise}
Figure~\ref{email50NumCharHist} reveals only one prominent mode in the number of characters. Is the distribution unimodal, bimodal, or multimodal?\footnote{Unimodal. Remember that \emph{uni} stands for 1 (think \emph{uni}cycles). Similarly, \emph{bi} stands for~2 (think \emph{bi}cycles). (We're hoping a \emph{multicycle} will be invented to complete this analogy.)}
\end{exercise}

\begin{exercise}
Height measurements of young students and adult teachers at a K-3 elementary school were taken. How many modes would you anticipate in this height data set?\footnote{There might be two height groups visible in the data set: one of the students and one of the adults. That is, the data are probably bimodal.}
\end{exercise}

\begin{tipBox}{\tipBoxTitle{Looking for modes}
Looking for modes isn't about finding a clear and correct answer about the number of modes in a distribution, which is why \emph{prominent} is not rigorously defined in this book. The important part of this examination is to better understand your data and how it might be structured.}
\end{tipBox}


\subsection{Variance and standard deviation}
\label{variability}

The mean was introduced as a method to describe the center of a data set, but the \indexthis{variability}{variability} in the data is also important. Here, we introduce two measures of variability: the variance and the standard deviation. Both of these are very useful in data analysis, even though their formulas are a bit tedious to calculate by hand. The standard deviation is the easier of the two to understand, and it roughly describes how far away the typical observation is from the mean.

We call the distance of an observation from its mean its \term{deviation}. Below are the deviations for the $1^{st}_{}$, $2^{nd}_{}$, $3^{rd}$, and $50^{th}_{}$ observations in the \var{num\_\hspace{0.3mm}char} variable. For computational convenience, the number of characters is listed in the thousands and rounded to the first decimal.
\begin{align*}
x_1^{}-\bar{x} &= 21.7 - 11.6 = 10.1 \hspace{5mm}\text{ } \\
x_2^{}-\bar{x} &= 7.0 - 11.6 = -4.6 \\
x_3^{}-\bar{x} &= 0.6 - 11.6 = -11.0 \\
			&\ \vdots \\
x_{50}^{}-\bar{x} &= 15.8 - 11.6 = 4.2
\end{align*}
% library(openintro); d <- email50$num_char; round(mean(d),1); d[c(1,2,3,50)]; d[c(1,2,3,50)] - round(mean(d),1); (d[c(1,2,3,50)] - round(mean(d)))^2; sum((d - round(mean(d)))^2)/49; sqrt(sum((d - round(mean(d)))^2)/49); var(d); sd(d)
If we square these deviations and then take an average, the result is about equal to the sample \term{variance}\label{varianceIsDefined}, denoted by $s_{}^2$\marginpar[\raggedright$s^2_{}$\\\footnotesize sample variance]{\raggedright$s^2_{}$\\\footnotesize sample variance}:
\begin{align*}
s_{}^2 &= \frac{10.1_{}^2 + (-4.6)_{}^2 + (-11.0)_{}^2 + \cdots + 4.2_{}^2}{50-1} \\
	&= \frac{102.01 + 21.16 + 121.00 + \cdots + 17.64}{49} \\
	&= 172.44
\end{align*}
We divide by $n-1$, rather than dividing by $n$, when computing the variance; you need not worry about this mathematical nuance for the material in this textbook. Notice that squaring the deviations does two things. First, it makes large values much larger, seen by comparing $10.1^2$, $(-4.6)^2$, $(-11.0)^2$, and $4.2^2$. Second, it gets rid of any negative signs.

The \term{standard deviation} is defined as the square root of the variance:
$$s=\sqrt{172.44} = 13.13$$
\marginpar[\raggedright\vspace{-10mm}

$s$\\\footnotesize sample standard deviation]{\raggedright\vspace{-10mm}

$s$\\\footnotesize sample standard deviation
}\index{s@$s$}The standard deviation of the number of characters in an email is about 13.13 thousand. A subscript of $_x$ may be added to the variance and standard deviation, i.e. $s_x^2$ and $s_x^{}$, as a reminder that these are the variance and standard deviation of the observations represented by $x_1^{}$, $x_2^{}$, ..., $x_n^{}$. The $_{x}$ subscript is usually omitted when it is clear which data the variance or standard deviation is referencing.

\begin{termBox}{\tBoxTitle{Variance and standard deviation}
The variance is roughly the average squared distance from the mean. The standard deviation is the square root of the variance. The standard deviation is useful when considering how close the data are to the mean.}
\end{termBox}

Formulas and methods used to compute the variance and standard deviation for a population are similar to those used for a sample.\footnote{The only difference is that the population variance has a division by $n$ instead of $n-1$.} However, like the mean, the population values have special symbols: $\sigma_{}^2$\marginpar[\raggedright$\sigma_{}^2$\\\footnotesize population variance\\ \hspace{2mm}]{\raggedright$\sigma_{}^2$\\\footnotesize population variance\\ \hspace{2mm}} for the variance and $\sigma$\marginpar[\raggedright$\sigma$\\\footnotesize population standard deviation\\ \hspace{2mm}]{\raggedright$\sigma$\\\footnotesize population standard deviation\\ \hspace{2mm}} for the standard deviation. The symbol $\sigma$ \index{Greek!sigma@sigma ($\sigma$)} is the Greek letter \emph{sigma}.


\begin{figure}
\centering
\includegraphics[width=\mycaptionwidth]{ch_intro_to_data/figures/sdAsRuleForEmailNumChar/sdAsRuleForEmailNumChar}
\caption{In the \var{num\_\hspace{0.3mm}char} data, 41 of the 50 emails (82\%) are within 1~standard deviation of the mean, and 47 of the 50 emails (94\%) are within 2 standard deviations. Usually about 70\% of the data are within 1 standard deviation of the mean and 95\% are within 2 standard deviations, though this rule of thumb is less accurate for skewed data, as shown in this example.}
\label{sdAsRuleForEmailNumChar}
\end{figure}

\begin{tipBox}{\tipBoxTitle{standard deviation describes variability}
Focus on the conceptual meaning of the standard deviation as a descriptor of variability rather than the formulas. Usually 70\% of the data will be within one standard deviation of the mean and about 95\% will be within two standard deviations. However, as seen in Figures~\ref{sdAsRuleForEmailNumChar} and~\ref{severalDiffDistWithSdOf1}, these percentages are not strict rules.}
\end{tipBox}

\begin{figure}
\centering
\includegraphics[width=0.64\textwidth]{ch_intro_to_data/figures/severalDiffDistWithSdOf1/severalDiffDistWithSdOf1}
\caption{Three very different population distributions with the same mean $\mu=0$ and standard deviation $\sigma=1$.}
\label{severalDiffDistWithSdOf1}
\end{figure}

\begin{exercise}
On page~\pageref{shapeFirstDiscussed}, the concept of shape of a distribution was introduced. A good description of the shape of a distribution should include modality and whether the distribution is symmetric or skewed to one side. Using Figure~\ref{severalDiffDistWithSdOf1} as an example, explain why such a description is important.\footnote{Figure~\ref{severalDiffDistWithSdOf1} shows three distributions that look quite different, but all have the same mean, variance, and standard deviation. Using modality, we can distinguish between the first plot (bimodal) and the last two (unimodal). Using skewness, we can distinguish between the last plot (right skewed) and the first two. While a picture, like a histogram, tells a more complete story, we can use modality and shape (symmetry/skew) to characterize basic information about a~distribution.}
\end{exercise}

\begin{example}{Describe the distribution of the \var{num\_\hspace{0.3mm}char} variable using the histogram in Figure~\vref{email50NumCharHist}. The description should incorporate the center, variability, and shape of the distribution, and it should also be placed in context: the number of characters in emails. Also note any especially unusual cases.}
The distribution of email character counts is unimodal and very strongly skewed to the high end. Many of the counts fall near the mean at 11,600, and most fall within one standard deviation (13,130) of the mean. There is one exceptionally long email with about 65,000 characters.
\end{example}

In practice, the variance and standard deviation are sometimes used as a means to an end, where the ``end'' is being able to accurately estimate the uncertainty associated with a sample statistic. For example, in Chapter~\ref{foundationsForInference} we will use the variance and standard deviation to assess how close the sample mean is to the population mean.

\subsection{Box plots, quartiles, and the median}

A \term{box plot} summarizes a data set using five statistics while also plotting unusual observations. Figure~\ref{boxPlotLayoutNumVar} provides a vertical dot plot alongside a box plot of the \var{num\_\hspace{0.3mm}char} variable from the \data{email50} data set.

\begin{figure}[h]
   \centering
   \includegraphics[width=0.86\mycaptionwidth]{ch_intro_to_data/figures/boxPlotLayoutNumVar/boxPlotLayoutNumVar}
   \caption{A vertical dot plot next to a labeled box plot for the number of characters in 50 emails. The median (6,890), splits the data into the bottom 50\% and the top 50\%, marked in the dot plot by horizontal dashes and open circles, respectively.}
   \label{boxPlotLayoutNumVar}
\end{figure}

The first step in building a box plot is drawing a dark line denoting the \term{median}, which splits the data in half. Figure~\ref{boxPlotLayoutNumVar} shows 50\% of the data falling below the median (dashes) and other 50\% falling above the median (open circles). There are 50 character counts in the data set (an even number) so the data are perfectly split into two groups of~25. We take the median in this case to be the average of the two observations closest to the $50^{th}$ percentile: $(\text{6,768} + \text{7,012}) / 2 = \text{6,890}$. When there are an odd number of observations, there will be exactly one observation that splits the data into two halves, and in this case that observation is the median (no average needed).

\begin{termBox}{\tBoxTitle{Median: the number in the middle}
If the data are ordered from smallest to largest, the \term{median} is the observation right in the middle. If there are an even number of observations, there will be two values in the middle, and the median is taken as their average.}
\end{termBox}

The second step in building a box plot is drawing a rectangle to represent the middle 50\% of the data. The total length of the box, shown vertically in Figure~\ref{boxPlotLayoutNumVar}, is called the \term{interquartile range} (\hiddenterm{IQR}, for short). It, like the standard deviation, is a measure of \indexthis{variability}{variability} in data. The more variable the data, the larger the standard deviation and~IQR. The two boundaries of the box are called the \term{first quartile} \index{quartile!first quartile} (the $25^{th}$ \hiddenterm{percentile}, i.e. 25\% of the data fall below this value) and the \term{third quartile} \index{quartile!third quartile} (the $75^{th}$ percentile), and these are often labeled $Q_1$ \index{Q$_1$} and $Q_3$\index{Q$_3$}, respectively.

\begin{termBox}{\tBoxTitle{Interquartile range (IQR)}
The IQR\index{interquartile range} is the length of the box in a box plot. It is computed as
\begin{eqnarray*}
IQR = Q_3 - Q_1
\end{eqnarray*}
where $Q_1$ and $Q_3$ are the $25^{th}$ and $75^{th}$ percentiles.}
\end{termBox}

\begin{exercise}
What percent of the data fall between $Q_1$ and the median? What percent is between the median and $Q_3$?\footnote{Since $Q_1$ and $Q_3$ capture the middle 50\% of the data and the median splits the data in the middle, 25\% of the data fall between $Q_1$ and the median, and another 25\% falls between the median and $Q_3$.}
\end{exercise}

Extending out from the box, the \term{whiskers} attempt to capture the data outside of the box, however, their reach is never allowed to be more than $1.5\times IQR$.\footnote{While the choice of exactly 1.5 is arbitrary, it is the most commonly used value for box plots.} They capture everything within this reach. In Figure~\ref{boxPlotLayoutNumVar}, the upper whisker does not extend to the last three points, which is beyond $Q_3 + 1.5\times IQR$, and so it extends only to the last point below this limit. The lower whisker stops at the lowest value, 33, since there is no additional data to reach; the lower whisker's limit is not shown in the figure because the plot does not extend down to $Q_1 - 1.5\times IQR$. In a sense, the box is like the body of the box plot and the whiskers are like its arms trying to reach the rest of the data.

Any observation that lies beyond the whiskers is labeled with a dot. The purpose of labeling these points -- instead of just extending the whiskers to the minimum and maximum observed values -- is to help identify any observations that appear to be unusually distant from the rest of the data. Unusually distant observations are called \termsub{outliers}{outlier}. In this case, it would be reasonable to classify the emails with character counts of 41,623, 42,793, and 64,401 as outliers since they are numerically distant from most of the data.

\begin{termBox}{\tBoxTitle{Outliers are extreme}
An \term{outlier} is an observation that appears extreme relative to the rest of the data.}
\end{termBox}


\begin{tipBox}{\tipBoxTitle{Why it is important to look for outliers}
Examination of data for possible outliers serves many useful purposes, including\vspace{-2mm}
\begin{enumerate}
\setlength{\itemsep}{0mm}
\item Identifying \indexthis{strong skew}{skew!example: strong} in the distribution.
\item Identifying data collection or entry errors. For instance, we re-examined the email purported to have 64,401 characters to ensure this value was accurate.
\item Providing insight into interesting properties of the data.\vspace{0.5mm}
\end{enumerate}}
\end{tipBox}

\begin{exercise}
The observation 64,401, a suspected outlier, was found to be an accurate observation. What would such an observation suggest about the nature of character counts in emails?\footnote{That occasionally there may be very long emails.}\end{exercise}

\begin{exercise}
Using Figure~\ref{boxPlotLayoutNumVar}, estimate the following values for \var{num\_\hspace{0.3mm}char} in the \data{email50} data set: (a) $Q_1$, (b) $Q_3$, and (c) IQR.\footnote{These visual estimates will vary a little from one person to the next: $Q_1=$ 3,000, $Q_3=$ 15,000, $\text{IQR}=Q_3 - Q_1 = $ 12,000. (The true values: $Q_1=$ 2,536, $Q_3=$ 15,411, $\text{IQR} = $ 12,875.)}
\end{exercise}

\CalculatorVideos{how to create statistical summaries and box plots}


\subsection{Robust statistics}

How are the \indexthis{sample statistics}{sample statistic} of the \data{num\_\hspace{0.3mm}char} data set affected by the observation, 64,401? What would have happened if this email wasn't observed? What would happen to these \indexthis{summary statistics}{summary statistic} if the observation at 64,401 had been even larger, say 150,000? These scenarios are plotted alongside the original data in Figure~\ref{email50NumCharDotPlotRobustEx}, and sample statistics are computed under each scenario in Table~\ref{robustOrNotTable}.

\begin{figure}[ht]
\centering
\includegraphics[width=\textwidth]{ch_intro_to_data/figures/email50NumCharDotPlotRobustEx/email50NumCharDotPlotRobustEx}
\caption{Dot plots of the original character count data and two modified data sets.}
\label{email50NumCharDotPlotRobustEx}
\end{figure}

\begin{table}[ht]
\centering
\begin{tabular}{l c cc c cc}
  \hline
& \hspace{0mm} & \multicolumn{2}{c}{\bf robust} & \hspace{2mm} & \multicolumn{2}{c}{\bf not robust} \\
scenario && median & IQR && $\bar{x}$ & $s$ \\
  \hline
original \var{num\_\hspace{0.3mm}char} data 	&& 6,890 & 12,875 && 11,600 & 13,130 \\
% library(openintro); data(email50); d <- email50$num_char; median(d); diff(quantile(d, c(0.25,0.75))); mean(d); sd(d)
drop 64,401 observation		&& 6,768 & 11,702 && 10,521 & 10,798 \\
% library(openintro); data(email50); d <- email50$num_char; d <- d[-which.max(d)]; median(d); diff(quantile(d, c(0.25,0.75))); mean(d); sd(d)
move 64,401 to 150,000		&& 6,890 & 12,875 && 13,310 & 22,434 \\
% library(openintro); data(email50); d <- email50$num_char; d[which.max(d)] <- 100000; median(d); diff(quantile(d, c(0.25,0.75))); mean(d); sd(d)
   \hline
\end{tabular}
\caption{A comparison of how the median, IQR, mean ($\bar{x}$), and standard deviation ($s$) change when extreme observations are present.}
\label{robustOrNotTable}
\end{table}

\begin{exercise} \label{numCharWhichIsMoreRobust}
(a) Which is more affected by extreme observations, the mean or median? Table~\ref{robustOrNotTable} may be helpful. (b) Is the standard deviation or IQR more affected by extreme observations?\footnote{(a) Mean is affected more. (b) Standard deviation is affected more. Complete explanations are provided in the material following Guided Practice~\ref{numCharWhichIsMoreRobust}.}
\end{exercise}

The median and IQR are called \term{robust estimates} because extreme observations have little effect on their values. The mean and standard deviation are much more affected by changes in extreme observations.

\begin{example}{The median and IQR do not change much under the three scenarios in Table~\ref{robustOrNotTable}. Why might this be the case?}
The median and IQR are only sensitive to numbers near $Q_1$, the median, and $Q_3$. Since values in these regions are relatively stable -- there aren't large jumps between observations -- the median and IQR estimates are also quite stable.
\end{example}

\begin{exercise}
The distribution of vehicle prices tends to be right skewed, with a few luxury and sports cars lingering out into the right tail. If you were searching for a new car and cared about price, should you be more interested in the mean or median price of vehicles sold, assuming you are in the market for a regular car?\footnote{Buyers of a ``regular car'' should be concerned about the median price. High-end car sales can drastically inflate the mean price while the median will be more robust to the influence of those sales.}
\end{exercise}

\subsection{Transforming data (special topic)}
\label{transformingDataSubsection}

When data are very strongly skewed, we sometimes transform them so they are easier to model. Consider the histogram of salaries for Major League Baseball players' salaries from 2010, which is shown in Figure~\ref{histMLBSalariesReg}.

\begin{figure}[ht]
\centering
\subfigure[]{
\includegraphics[width=0.46\textwidth]{ch_intro_to_data/figures/histMLBSalaries/histMLBSalariesReg}
\label{histMLBSalariesReg}
}
\subfigure[]{
\includegraphics[width=0.46\textwidth]{ch_intro_to_data/figures/histMLBSalaries/histMLBSalariesLog}
\label{histMLBSalariesLog}
}
\caption{\subref{histMLBSalariesReg} Histogram of MLB player salaries for 2010, in millions of dollars. \subref{histMLBSalariesLog} Histogram of the log-transformed MLB player salaries for 2010.}
\label{histMLBSalaries}
\end{figure}

\begin{example}{The histogram of MLB player salaries is useful in that we can see the data are extremely skewed\index{skew!example: extreme} and centered (as gauged by the median) at about \$1 million. What isn't useful about this plot?}
Most of the data are collected into one bin in the histogram and the data are so strongly skewed that many details in the data are obscured.
\end{example}

There are some standard transformations that are often applied when much of the data cluster near zero (relative to the larger values in the data set) and all observations are positive. A \term{transformation} is a rescaling of the data using a function. For instance, a plot of the natural logarithm\footnote{Statisticians often write the natural logarithm as $\log$. You might be more familiar with it being written as $\ln$.} of player salaries results in a new histogram in Figure~\ref{histMLBSalariesLog}. Transformed data are sometimes easier to work with when applying statistical models because the transformed data are much less skewed and outliers are usually less extreme.

Transformations can also be applied to one or both variables in a scatterplot. A scatterplot of the \var{line\_\hspace{0.3mm}breaks} and \var{num\_\hspace{0.3mm}char} variables is shown in Figure~\ref{email50LinesCharactersMod}, which was earlier shown in Figure~\ref{email50LinesCharacters}. We can see a positive association between the variables and that many observations are clustered near zero. In Chapter~\ref{linRegrForTwoVar}, we might want to use a straight line to model the data. However, we'll find that the data in their current state cannot be modeled very well. Figure~\ref{email50LinesCharactersModLog} shows a scatterplot where both the \var{line\_\hspace{0.3mm}breaks} and \var{num\_\hspace{0.3mm}char} variables have been transformed using a log (base $e$) transformation. While there is a positive association in each plot, the transformed data show a steadier trend, which is easier to model than the untransformed data.

\begin{figure}
\centering
\subfigure[]{
\includegraphics[width=0.47\textwidth]{ch_intro_to_data/figures/email50LinesCharactersMod/email50LinesCharactersMod}
\label{email50LinesCharactersMod}
}
\subfigure[]{
\includegraphics[width=0.47\textwidth]{ch_intro_to_data/figures/email50LinesCharactersMod/email50LinesCharactersModLog}
\label{email50LinesCharactersModLog}
}
\caption{\subref{email50LinesCharactersMod} Scatterplot of \var{line\_\hspace{0.3mm}breaks} against \var{num\_\hspace{0.3mm}char} for 50 emails. \subref{email50LinesCharactersModLog} A scatterplot of the same data but where each variable has been log-transformed.}
\label{email50LinesCharactersModMain}
\end{figure}

Transformations other than the logarithm can be useful, too. For instance, the square root ($\sqrt{\text{original observation}}$) and inverse ($\frac{1}{\text{original observation}}$) are used by statisticians. Common goals in transforming data are to see the data structure differently, reduce skew, assist in modeling, or straighten a nonlinear relationship in a scatterplot.

\index{data!email50|)}

\subsection{Mapping data (special topic)}

\index{data!county|(}
\index{intensity map|(}

The \data{county} data set offers many numerical variables that we could plot using dot plots, scatterplots, or box plots, but these miss the true nature of the data. Rather, when we encounter geographic data, we should map it using an \term{intensity map}, where colors are used to show higher and lower values of a variable. Figures~\ref{countyIntensityMaps1} and~\ref{countyIntensityMaps2} shows intensity maps for federal spending per capita (\var{fed\_\hspace{0.3mm}spend}), poverty rate in percent (\var{poverty}), homeownership rate in percent (\var{homeownership}), and median household income (\var{med\_\hspace{0.3mm}income}). The color key indicates which colors correspond to which values. Note that the intensity maps are not generally very helpful for getting precise values in any given county, but they are very helpful for seeing geographic trends and generating interesting research questions.

\begin{figure}
\centering
\subfigure[]{\includegraphics[width=\textwidth]{ch_intro_to_data/figures/countyIntensityMaps/countyFedSpendMap}\label{countyFedSpendMap}}
\subfigure[]{\includegraphics[width=\textwidth]{ch_intro_to_data/figures/countyIntensityMaps/countyPovertyMap}\label{countyPovertyMap}}
\caption{\subref{countyFedSpendMap} Map of federal spending (dollars per capita). \subref{countyPovertyMap} Intensity map of poverty rate (percent).}
\label{countyIntensityMaps1}
\end{figure}

\begin{figure}
\centering
\subfigure[]{\includegraphics[width=\textwidth]{ch_intro_to_data/figures/countyIntensityMaps/countyHomeownershipMap}\label{countyHomeownershipMap}}
\subfigure[]{\includegraphics[width=\textwidth]{ch_intro_to_data/figures/countyIntensityMaps/countyMedIncomeMap}\label{countyMedIncomeMap}}
\caption{\subref{countyHomeownershipMap} Intensity map of homeownership rate (percent). \subref{countyMedIncomeMap} Intensity map of median household income (\$1000s).}
\label{countyIntensityMaps2}
\end{figure}

\textC{\pagebreak}

\begin{example}{What interesting features are evident in the \var{fed\_\hspace{0.3mm}spend} and \var{poverty} intensity maps?}
The federal spending intensity map shows substantial spending in the Dakotas and along the central-to-western part of the Canadian border, which may be related to the oil boom in this region. There are several other patches of federal spending, such as a vertical strip in eastern Utah and Arizona and the area where Colorado, Nebraska, and Kansas meet. There are also seemingly random counties with very high federal spending relative to their neighbors. If we did not cap the federal spending range at \$18 per capita, we would actually find that some counties have extremely high federal spending while there is almost no federal spending in the neighboring counties. These high-spending counties might contain military bases, companies with large government contracts, or other government facilities with many employees.

Poverty rates are evidently higher in a few locations. Notably, the deep south shows higher poverty rates, as does the southwest border of Texas. The vertical strip of eastern Utah and Arizona, noted above for its higher federal spending, also appears to have higher rates of poverty (though generally little correspondence is seen between the two variables).  High poverty rates are evident in the Mississippi flood plains a little north of New Orleans and also in a large section of Kentucky and West Virginia.
\end{example}

\begin{exercise}
What interesting features are evident in the \var{med\_\hspace{0.3mm}income} intensity map in Figure~\ref{countyMedIncomeMap}?\footnote{Note: answers will vary. There is a very strong correspondence between high earning and metropolitan areas. You might look for large cities you are familiar with and try to spot them on the map as dark spots.}
\end{exercise}

\index{intensity map|)}
\index{data!county|)}



\textC{\newpage}

\section[Considering categorical data]{Considering categorical data \sectionvideohref{youtube-7NhNeADL8fA&list=PLkIselvEzpM6pZ76FD3NoCvvgkj_p-dE8}}
\label{categoricalData}

\index{data!email|(}

Like numerical data, categorical data can also be organized and analyzed. In this section, we will introduce tables and other basic tools for categorical data that are used throughout this book. The \data{email50} data set represents a sample from a larger email data set called \data{email}. This larger data set contains information on 3,921 emails. In this section we will examine whether the presence of numbers, small or large, in an email provides any useful value in classifying email as spam or not spam.
% library(openintro); data(email); dim(email)

\subsection{Contingency tables and bar plots}

Table~\ref{emailSpamNumberTableTotals} summarizes two variables: \var{spam} and \var{number}. Recall that \var{number} is a categorical variable that describes whether an email contains no numbers, only small numbers (values under 1 million), or at least one big number (a value of 1 million or more). A table that summarizes data for two categorical variables in this way is called a \term{contingency table}. Each value in the table represents the number of times a particular combination of variable outcomes occurred. For example, the value 149 corresponds to the number of emails in the data set that are spam \emph{and} had no number listed in the email. Row and column totals are also included. The \term{row totals} \index{contingency table!row totals} provide the total counts across each row (e.g. $149 + 168 + 50 = 367$), and \term{column totals} \index{contingency table!column totals} are total counts down each column.

A table for a single variable is called a \term{frequency table}. Table~\ref{emailNumberTable} is a frequency table for the \var{number} variable. If we replaced the counts with percentages or proportions, the table would be called a \term{relative frequency table}.

\begin{table}[ht]
\centering
\begin{tabular}{ll  ccc  rr}
& & \multicolumn{3}{c}{\bf \var{number}} & \\
  \cline{3-5}
& & none & small & big & Total & \hspace{2mm}\  \\
  \cline{2-6}
	 & spam &  149 & 168 &  50 & 367 \\
\raisebox{1.5ex}[0pt]{\var{spam}}
	& not spam &  400 & 2659 & 495 & 3554 \\
  \cline{2-6}
& Total & 549 & 2827 & 545 & 3921 \\
  \cline{2-6}
\end{tabular}
\caption{A contingency table for \var{spam} and \var{number}.}
\label{emailSpamNumberTableTotals}
%library(openintro); library(xtable); data(email); tab <- table(email[,c("spam", "number")])[2:1,]; xtable(tab); rowSums(tab); colSums(tab); sum(tab)
\end{table}

\begin{table}[htb]
\centering
\begin{tabular}{cccc}
  \hline
none & small & big & Total \\
 % \hline
 549 & 2827 & 545 & 3921 \\
   \hline
\end{tabular}
\caption{A frequency table for the \var{number} variable.}
\label{emailNumberTable}
\end{table}
%library(openintro); library(xtable); data(email); xtable(table(email[,c("html")]))

A bar plot is a common way to display a single categorical variable. The left panel of Figure~\ref{emailNumberBarPlot} shows a \term{bar plot} for the \var{number} variable. In the right panel, the counts are converted into proportions (e.g. $549/3921=0.140$ for \resp{none}), showing the proportion of observations that are in each level (i.e. in each category).

\begin{figure}[bht]
   \centering
   \includegraphics[width=0.9\textwidth]{ch_intro_to_data/figures/emailNumberBarPlot/emailNumberBarPlot}
   \caption{Two bar plots of \var{number}. The left panel shows the counts, and the right panel shows the proportions in each group.}
   \label{emailNumberBarPlot}
\end{figure}


\subsection{Row and column proportions}

Table~\ref{rowPropSpamNumber} shows the row proportions for Table~\ref{emailSpamNumberTableTotals}. The \termsub{row proportions}{contingency table!row proportions} are computed as the counts divided by their row totals. The value 149 at the intersection of \resp{spam} and \resp{none} is replaced by $149/367=0.406$, i.e. 149 divided by its row total, 367. So what does 0.406 represent? It corresponds to the proportion of spam emails in the sample that do not have any numbers.

\begin{table}
\centering
\begin{tabular}{l rrr r}
  \hline
 & none & small & big & Total \\
  \hline
spam &  $149/367 = 0.406$ & $168/367 = 0.458$ &
			$50/367 = 0.136$ & 1.000 \\
not spam &  $400/3554 = 0.113$ & $2657/3554 = 0.748$ &
			$495/3554 = 0.139$ & 1.000 \\
   \hline
Total & $549/3921 = 0.140$ & $2827/3921 = 0.721$ &
			$545/3921 = 0.139$ & 1.000 \\
  \hline
\end{tabular}
\caption{A contingency table with row proportions for the \var{spam} and \var{number} variables.}
\label{rowPropSpamNumber}
% library(openintro); data(email); g <- table(email$spam, email$number)[2:1,]; g / rep(rowSums(g), 3); rowSums(g)
\end{table}

A contingency table of the column proportions is computed in a similar way, where each \termsub{column proportion}{contingency table!column proportion} is computed as the count divided by the corresponding column total. Table~\ref{colPropSpamNumber} shows such a table, and here the value 0.271 indicates that 27.1\% of emails with no numbers were spam. This rate of spam is much higher compared to emails with only small numbers (5.9\%) or big numbers (9.2\%). Because these spam rates vary between the three levels of \var{number} (\resp{none}, \resp{small}, \resp{big}), this provides evidence that the \var{spam} and \var{number} variables are associated.

\begin{table}[h]
\centering\small
\begin{tabular}{l rrr r}
  \hline
 & none & small & big & Total \\
  \hline
spam &  $149/549 = 0.271$ & $168/2827 = 0.059$ &
				$50/545 = 0.092$ & $367/3921 = 0.094$ \\
not spam &  $400/549 = 0.729$ & $2659/2827 = 0.941$ &
				$495/545 = 0.908$ & $3684/3921 = 0.906$ \\
   \hline
Total & 1.000 & 1.000 & 1.000 & 1.000 \\
   \hline
\end{tabular}
\caption{A contingency table with column proportions for the \var{spam} and \var{number} variables.}
\label{colPropSpamNumber}
% library(openintro); data(email); g <- table(email$spam, email$number)[2:1,]; g / rep(colSums(g), rep(2, 3)); g; colSums(g)
\end{table}

We could also have checked for an association between \var{spam} and \var{number} in Table~\ref{rowPropSpamNumber} using row proportions. When comparing these row proportions, we would look down columns to see if the fraction of emails with no numbers, small numbers, and big numbers varied from \resp{spam} to \resp{not~spam}.

\begin{exercise}
What does 0.458 represent in Table~\ref{rowPropSpamNumber}? What does 0.059 represent in Table~\ref{colPropSpamNumber}?\footnote{0.458 represents the proportion of spam emails that had a small number. 0.059 represents the fraction of emails with small numbers that are spam.}
\end{exercise}

\begin{exercise}
What does 0.139 at the intersection of \resp{not~spam} and \resp{big} represent in Table~\ref{rowPropSpamNumber}? What does 0.908 represent in the Table~\ref{colPropSpamNumber}?\footnote{0.139 represents the fraction of non-spam email that had a big number. 0.908 represents the fraction of emails with big numbers that are non-spam emails.}
\end{exercise}

\begin{example}{Data scientists use statistics to filter spam from incoming email messages. By noting specific characteristics of an email, a data scientist may be able to classify some emails as spam or not spam with high accuracy. One of those characteristics is whether the email contains no numbers, small numbers, or big numbers. Another characteristic is whether or not an email has any HTML content. A contingency table for the \var{spam} and \var{format} variables from the \data{email} data set are shown in Table~\ref{emailSpamHTMLTableTotals}. Recall that an HTML email is an email with the capacity for special formatting, e.g. bold text. In Table~\ref{emailSpamHTMLTableTotals}, which would be more helpful to someone hoping to classify email as spam or regular email: row or column proportions?} \label{weighingRowColumnProportions}
Such a person would be interested in how the proportion of spam changes within each email format. This corresponds to column proportions: the proportion of spam in plain text emails and the proportion of spam in HTML emails.

If we generate the column proportions, we can see that a higher fraction of plain text emails are spam ($209/1195 = 17.5\%$) than compared to HTML emails ($158/2726 = 5.8\%$). This information on its own is insufficient to classify an email as spam or not spam, as over 80\% of plain text emails are not spam. Yet, when we carefully combine this information with many other characteristics, such as \var{number} and other variables, we stand a reasonable chance of being able to classify some email as spam or not spam. \GLMSection{This is a topic we will return to in Chapter~\ref{multipleRegressionAndANOVA}.}{}
\end{example}

\begin{table}[ht]
\centering
\begin{tabular}{l cc r}
  \hline
 & text & HTML & Total \\
  \hline
spam & 209 & 158 & 367 \\
not spam & 986 & 2568 & 3554 \\
   \hline
Total & 1195 & 2726 & 3921 \\
   \hline
\end{tabular}
\caption{A contingency table for \var{spam} and \var{format}.}
\label{emailSpamHTMLTableTotals}
%library(openintro); library(xtable); data(email); tab <- table(email[,c("spam", "format")])[2:1,]; tab; colSums(tab); rowSums(tab)
\end{table}

Example~\ref{weighingRowColumnProportions} points out that row and column proportions are not equivalent. Before settling on one form for a table, it is important to consider each to ensure that the most useful table is constructed.

\begin{exercise}
Look back to Tables~\ref{rowPropSpamNumber} and~\ref{colPropSpamNumber}. Which would be more useful to someone hoping to identify spam emails using the \var{number} variable?\footnote{The column proportions in Table~\ref{colPropSpamNumber} will probably be most useful, which makes it easier to see that emails with small numbers are spam about 5.9\% of the time (relatively rare). We would also see that about 27.1\% of emails with no numbers are spam, and 9.2\% of emails with big numbers are spam.}
\end{exercise}


\textC{\newpage}

\subsection{Segmented bar and mosaic plots}
\label{segmentedBarPlotsAndIndependence}

Contingency tables using row or column proportions are especially useful for examining how two categorical variables are related. Segmented bar and mosaic plots provide a way to visualize the information in these tables.

A \termsub{segmented bar plot}{bar plot!segmented bar plot} is a graphical display of contingency table information. For example, a segmented bar plot representing Table~\ref{colPropSpamNumber} is shown in Figure~\ref{emailSpamNumberSegBar}, where we have first created a bar plot using the \var{number} variable and then divided each group by the levels of \var{spam}. The column proportions of Table~\ref{colPropSpamNumber} have been translated into a standardized segmented bar plot in Figure~\ref{emailSpamNumberSegBarSta}, which is a helpful visualization of the fraction of spam emails in each level of \var{number}.

\begin{figure}[h]
\centering
\subfigure[]{
\includegraphics[width=0.46\textwidth]{ch_intro_to_data/figures/emailSpamNumberSegBar/emailSpamNumberSegBar}
\label{emailSpamNumberSegBar}
}
\subfigure[]{
\includegraphics[width=0.46\textwidth]{ch_intro_to_data/figures/emailSpamNumberSegBar/emailSpamNumberSegBarSta}
\label{emailSpamNumberSegBarSta}
}
\caption{\subref{emailSpamNumberSegBar} Segmented bar plot for numbers found in emails, where the counts have been further broken down by \var{spam}. \subref{emailSpamNumberSegBarSta} Standardized version of Figure~\subref{emailSpamNumberSegBar}.}
\label{emailSpamNumberSegBarPlot}
\end{figure}

\begin{example}{Examine both of the segmented bar plots. Which is more useful?}
Figure~\ref{emailSpamNumberSegBar} contains more information, but Figure~\ref{emailSpamNumberSegBarSta} presents the information more clearly. This second plot makes it clear that emails with no number have a relatively high rate of spam email -- about 27\%! On the other hand, less than 10\% of email with small or big numbers are spam.
\end{example}

Since the proportion of spam changes across the groups in Figure~\ref{emailSpamNumberSegBarSta}, we can conclude the variables are dependent, which is something we were also able to discern using table proportions. Because both the \resp{none} and \resp{big} groups have relatively few observations compared to the \resp{small} group, the association is more difficult to see in Figure~\ref{emailSpamNumberSegBar}.

In some other cases, a segmented bar plot that is not standardized will be more useful in communicating important information. Before settling on a particular segmented bar plot, create standardized and non-standardized forms and decide which is more effective at communicating features of the data.

\begin{figure}
\centering
\subfigure[]{
\includegraphics[width=0.3934\textwidth]{ch_intro_to_data/figures/emailSpamNumberMosaicPlot/emailNumberMosaic}
\label{emailNumberMosaic}
}
\subfigure[]{
\includegraphics[width=0.46\textwidth]{ch_intro_to_data/figures/emailSpamNumberMosaicPlot/emailSpamNumberMosaic}
\label{emailSpamNumberMosaic}
}
\caption{The one-variable mosaic plot for \var{number} and the two-variable mosaic plot for both \var{number} and \var{spam}.}
\label{emailSpamNumberMosaicPlot}
\end{figure}

A \term{mosaic plot} is a graphical display of contingency table information that is similar to a bar plot for one variable or a segmented bar plot when using two variables. Figure~\ref{emailNumberMosaic} shows a mosaic plot for the \var{number} variable. Each column represents a level of \var{number}, and the column widths correspond to the proportion of emails for each number~type. For~instance, there are fewer emails with no numbers than emails with only small numbers, so the no number email column is slimmer. In general, mosaic plots use box \emph{areas} to represent the number of observations that box represents.

\begin{figure}
   \centering
   \includegraphics[width=0.44\textwidth]{ch_intro_to_data/figures/emailSpamNumberMosaicPlot/emailSpamNumberMosaicRev}
   \caption{Mosaic plot where emails are grouped by the \var{number} variable after they've been divided into \resp{spam} and \resp{not spam}.}
   \label{emailSpamNumberMosaicRev}
\end{figure}

This one-variable mosaic plot is further divided into pieces in Figure~\ref{emailSpamNumberMosaic} using the \var{spam} variable. Each column is split proportionally according to the fraction of emails that were spam in each number category. For example, the second column, representing emails with only small numbers, was divided into emails that were spam (lower) and not spam (upper).
As another example, the bottom of the third column represents spam emails that had big numbers, and the upper part of the third column represents regular emails that had big numbers. We can again use this plot to see that the \var{spam} and \var{number} variables are associated since some columns are divided in different vertical locations than others, which was the same technique used for checking an association in the standardized version of the segmented bar plot.

In a similar way, a mosaic plot representing row proportions of Table~\ref{emailSpamNumberTableTotals} could be constructed, as shown in Figure~\ref{emailSpamNumberMosaicRev}. However, because it is more insightful for this application to consider the fraction of spam in each category of the \var{number} variable, we prefer Figure~\ref{emailSpamNumberMosaic}.

\subsection{The only pie chart you will see in this book}

While pie charts are well known, they are not typically as useful as other charts in a data analysis. A \term{pie chart} is shown in Figure~\vref{emailNumberPieChart} alongside a bar plot. It is generally more difficult to compare group sizes in a pie chart than in a bar plot, especially when categories have nearly identical counts or proportions. In the case of the \resp{none} and \resp{big} categories, the difference is so slight you may be unable to distinguish any difference in group sizes for either~plot!

\begin{figure}[h]
   \centering
   \includegraphics[width=\textwidth]{ch_intro_to_data/figures/emailNumberPieChart/emailNumberPieChart}
   \caption{A pie chart and bar plot of \var{number} for the \data{email} data set.}
   \label{emailNumberPieChart}
\end{figure}

\index{data!email|)}

\subsection{Comparing numerical data across groups}
\label{comparingAcrossGroups}

\index{data!county|(}

Some of the more interesting investigations can be considered by examining numerical data across groups. The methods required here aren't really new. All that is required is to make a numerical plot for each group. Here two convenient methods are introduced: side-by-side box plots and hollow histograms.

We will take a look again at the \data{county} data set and compare the median household income for counties that gained population from 2000 to 2010 versus counties that had no gain. While we might like to make a causal connection here, remember that these are observational data and so such an interpretation would be unjustified.

There were 2,041 counties where the population increased from 2000 to 2010, and there were 1,099 counties with no gain (all but one were a loss). A~random sample of 100 counties from the first group and 50 from the second group are shown in Table~\ref{countyIncomeSplitByPopGainTable} to give a better sense of some of the raw data.

\begin{table}
\centering
\begin{tabular}{ ccc ccc c ccc }
\multicolumn{6}{c}{\bf population gain} && \multicolumn{3}{c}{\bf no gain} \\
  \cline{1-6} \cline{8-10}
41.2 & 33.1 & 30.4 & 37.3 & 79.1 & 34.5 &\hspace{5mm}\ & 40.3 & 33.5 & 34.8 \\
22.9 & 39.9 & 31.4 & 45.1 & 50.6 & 59.4 && 29.5 & 31.8 & 41.3 \\
47.9 & 36.4 & 42.2 & 43.2 & 31.8 & 36.9 && 28 & 39.1 & 42.8 \\
50.1 & 27.3 & 37.5 & 53.5 & 26.1 & 57.2 && 38.1 & 39.5 & 22.3 \\
57.4 & 42.6 & 40.6 & 48.8 & 28.1 & 29.4 && 43.3 & 37.5 & 47.1 \\
43.8 & 26 & 33.8 & 35.7 & 38.5 & 42.3 && 43.7 & 36.7 & 36 \\
41.3 & 40.5 & 68.3 & 31 & 46.7 & 30.5 && 35.8 & 38.7 & 39.8 \\
68.3 & 48.3 & 38.7 & 62 & 37.6 & 32.2 && 46 & 42.3 & 48.2 \\
42.6 & 53.6 & 50.7 & 35.1 & 30.6 & 56.8 && 38.6 & 31.9 & 31.1 \\
66.4 & 41.4 & 34.3 & 38.9 & 37.3 & 41.7 && 37.6 & 29.3 & 30.1 \\
51.9 & 83.3 & 46.3 & 48.4 & 40.8 & 42.6 && 57.5 & 32.6 & 31.1 \\
44.5 & 34 & 48.7 & 45.2 & 34.7 & 32.2 && 46.2 & 26.5 & 40.1 \\
39.4 & 38.6 & 40 & 57.3 & 45.2 & 33.1 && 38.4 & 46.7 & 25.9 \\
43.8 & 71.7 & 45.1 & 32.2 & 63.3 & 54.7 && 36.4 & 41.5 & 45.7 \\
71.3 & 36.3 & 36.4 & 41 & 37 & 66.7 && 39.7 & 37 & 37.7 \\
50.2 & 45.8 & 45.7 & 60.2 & 53.1 &  && 21.4 & 29.3 & 50.1 \\
35.8 & 40.4 & 51.5 & 66.4 & 36.1 &  && 43.6 & 39.8 &  \\
\cline{1-6} \cline{8-10}
\end{tabular}
\caption{In this table, median household income (in \$1000s) from a random sample of 100 counties that gained population over 2000-2010 are shown on the left. Median incomes from a random sample of 50 counties that had no population gain are shown on the right.}
\label{countyIncomeSplitByPopGainTable}
\end{table}

The \term{side-by-side box plot} \index{box plot!side-by-side box plot} is a traditional tool for comparing across groups. An example is shown in the left panel of Figure~\ref{countyIncomeSplitByPopGain}, where there are two box plots, one for each group, placed into one plotting window and drawn on the same scale.

\begin{figure}
   \centering
   \includegraphics[width=\textwidth]{ch_intro_to_data/figures/countyIncomeSplitByPopGain/countyIncomeSplitByPopGain}
   \caption{Side-by-side box plot (left panel) and hollow histograms (right panel) for \var{med\_\hspace{0.3mm}income}, where the counties are split by whether there was a population gain or loss from 2000 to 2010. The income data were collected between 2006 and 2010.}
   \label{countyIncomeSplitByPopGain}
\end{figure}

Another useful plotting method uses \termsub{hollow histograms}{hollow histogram} to compare numerical data across groups. These are just the outlines of histograms of each group put on the same plot, as shown in the right panel of Figure~\ref{countyIncomeSplitByPopGain}.

\begin{exercise} \label{comparingPriceByTypeExercise}
Use the plots in Figure~\ref{countyIncomeSplitByPopGain} to compare the incomes for counties across the two groups. What do you notice about the approximate center of each group? What do you notice about the variability between groups? Is the shape relatively consistent between groups? How many \emph{prominent} modes are there for each group?\footnote{Answers may vary a little. The counties with population gains tend to have higher income (median of about \$45,000) versus counties without a gain (median of about \$40,000). The variability is also slightly larger for the population gain group. This is evident in the IQR, which is about 50\% bigger in the \emph{gain} group. Both distributions show slight to moderate right skew\index{skew!example: slight to moderate} and are unimodal. There is a secondary small bump at about \$60,000 for the \emph{no gain} group, visible in the hollow histogram plot, that seems out of place. (Looking into the data set, we would find that 8 of these 15 counties are in Alaska and Texas.) The box plots indicate there are many observations far above the median in each group, though we should anticipate that many observations will fall beyond the whiskers when using such a large data set.}
\end{exercise}

\begin{exercise}
What components of each plot in Figure~\ref{countyIncomeSplitByPopGain} do you find most useful?\footnote{Answers will vary. The side-by-side box plots are especially useful for comparing centers and spreads, while the hollow histograms are more useful for seeing distribution shape, skew, and groups of anomalies.}
\end{exercise}

\index{data!county|)}


%___________________________________________
\section[Case study: gender discrimination (special topic)]{Case study: gender discrimination \sectionvideohref{youtube-2pHhjx9hyM4&list=PLkIselvEzpM6pZ76FD3NoCvvgkj_p-dE8} \\(special topic)}
\label{caseStudyGenderDiscrimination}

\index{data!discrimination|(}

\begin{example}{Suppose your professor splits the students in class into two groups: students on the left and students on the right. If $\hat{p}_{_L}$ and $\hat{p}_{_R}$ represent the proportion of students who own an Apple product on the left and right, respectively, would you be surprised if $\hat{p}_{_L}$ did not {exactly} equal $\hat{p}_{_R}$?}\label{classRightLeftSideApple}
While the proportions would probably be close to each other, it would be unusual for them to be exactly the same. We would probably observe a small difference due to {chance}.
\end{example}

\begin{exercise}
If we don't think the side of the room a person sits on in class is related to whether the person owns an Apple product, what assumption are we making about the relationship between these two variables?\footnote{We would be assuming that these two variables are independent.}
\end{exercise}

\subsection{Variability within data}
\label{variabilityWithinData}

We consider a study investigating gender discrimination in the 1970s, which is set in the context of personnel decisions within a bank.\footnote{Rosen B and Jerdee T. 1974. Influence of sex role stereotypes on personnel decisions. Journal of Applied Psychology 59(1):9-14.} The research question we hope to answer is, ``Are females unfairly discriminated against in promotion decisions made by male managers?"

The participants in this study are 48 male bank supervisors attending a management institute at the University of North Carolina in 1972. They were asked to assume the role of the personnel director of a bank and were given a personnel file to judge whether the person should be promoted to a branch manager position. The files given to the participants were identical, except that half of them indicated the candidate was male and the other half indicated the candidate was female. These files were randomly assigned to the subjects.

\begin{exercise}
Is this an observational study or an experiment? What implications does the study type have on what can be inferred from the results?\footnote{The study is an experiment, as subjects were randomly assigned a male file or a female file. Since this is an experiment, the results can be used to evaluate a causal relationship between gender of a candidate and the promotion decision.}
\end{exercise}

For each supervisor we record the gender associated with the assigned file and the promotion decision. Using the results of the study summarized in Table~\ref{discriminationResults}, we would like to evaluate if females are unfairly discriminated against in promotion decisions. In this study, a smaller proportion of females are promoted than males (0.583 versus 0.875), but it is unclear whether the difference provides \emph{convincing evidence} that females are unfairly discriminated against.

\begin{table}[ht]
\centering
\begin{tabular}{l l cc rr}
& & \multicolumn{2}{c}{\var{decision}} \\
  \cline{3-4}
		&			& 	{promoted} 	& {not promoted} & Total & \hspace{3mm}  \\
  \cline{2-5}
		&	{male} 			& 21    		& 3   & 24  	 \\
  \raisebox{1.5ex}[0pt]{\var{gender}}		&	{female} 	& 14    		& 10     & 24	 \\
  \cline{2-5}
  		&	Total		& 35	& 13	&  48 \\
  \cline{2-5}
\end{tabular}
\caption{Summary results for the gender discrimination study.}
\label{discriminationResults}
\end{table}

\begin{example}{Statisticians are sometimes called upon to evaluate the strength of evidence. When looking at the rates of promotion for males and females in this study, what comes to mind as we try to determine whether the data show convincing evidence of a real difference?} \label{discriminationResultsWhatIsConvincingEvidence}
The observed promotion rates (58.3\% for females versus 87.5\% for males) suggest there might be discrimination against women in promotion decisions. However, we cannot be sure if the observed difference represents discrimination or is just from random chance. Generally there is a little bit of fluctuation in sample data, and we wouldn't expect the sample proportions to be \emph{exactly} equal, even if the truth was that the promotion decisions were independent of gender.
\end{example}

Example~\ref{discriminationResultsWhatIsConvincingEvidence} is a reminder that the observed outcomes in the sample may not perfectly reflect the true relationships between variables in the underlying population. Table~\ref{discriminationResults} shows there were 7 fewer promotions in the female group than in the male group, a difference in promotion rates of 29.2\% $\left( \frac{21}{24} - \frac{14}{24} = 0.292 \right)$. This difference is large, but the sample size for the study is small, making it unclear if this observed difference represents discrimination or whether it is simply due to chance. We label these two competing claims, $H_0$ and $H_A$:
\begin{itemize}
\setlength{\itemsep}{0mm}
\item[$H_0$:] \textbf{Independence model.} The variables \var{gender} and \var{decision} are independent. They have no relationship, and the observed difference between the proportion of males and females who were promoted, 29.2\%, was due to chance.
\item[$H_A$:] \textbf{Alternative model.} The variables \var{gender} and \var{decision} are \emph{not} independent. The difference in promotion rates of 29.2\% was not due to chance, and equally qualified females are less likely to be promoted than males.
\end{itemize}

What would it mean if the independence model, which says the variables \var{gender} and \var{decision} are unrelated, is true? It would mean each banker was going to decide whether to promote the candidate without regard to the gender indicated on the file. That~is, the difference in the promotion percentages was due to the way the files were randomly divided to the bankers, and the randomization just happened to give rise to a relatively large difference of 29.2\%.

Consider the alternative model: bankers were influenced by which gender was listed on the personnel file. If this was true, and especially if this influence was substantial, we would expect to see some difference in the promotion rates of male and female candidates. If this gender bias was against females, we would expect a smaller fraction of promotion decisions for female personnel files relative to the male files.

We choose between these two competing claims by assessing if the data conflict so much with $H_0$ that the independence model cannot be deemed reasonable. If this is the case, and the data support $H_A$, then we will reject the notion of independence and conclude there was discrimination.

\subsection{Simulating the study}
\label{simulatingTheStudy}

Table~\ref{discriminationResults} shows that 35 bank supervisors recommended promotion and 13 did not. Now, suppose the bankers' decisions were independent of gender. Then, if we conducted the experiment again with a different random arrangement of files, differences in promotion rates would be based only on random fluctuation. We can actually perform this \term{randomization}, which simulates what would have happened if the bankers' decisions had been independent of gender but we had distributed the files differently.

In this \term{simulation}, we thoroughly shuffle 48 personnel files, 24 labeled \resp{male\_\hspace{0.3mm}sim} and 24 labeled \resp{female\_\hspace{0.3mm}sim}, and deal these files into two stacks. We will deal 35 files into the first stack, which will represent the 35 supervisors who recommended promotion. The second stack will have 13 files, and it will represent the 13 supervisors who recommended against promotion. Then, as we did with the original data, we tabulate the results and determine the fraction of \resp{male\_\hspace{0.3mm}sim} and \resp{female\_\hspace{0.3mm}sim} who were promoted. The randomization of files in this simulation is independent of the promotion decisions, which means any difference in the two fractions is entirely due to chance. Table~\ref{discriminationRand1} show the results of such a simulation.

\begin{table}[ht]
\centering
\begin{tabular}{l l cc rr}
& & \multicolumn{2}{c}{\var{decision}} \\
  \cline{3-4}
		&			& 	{promoted} 	& {not promoted} & Total & \hspace{3mm}  \\
  \cline{2-5}
		&	\resp{male\_\hspace{0.3mm}sim} 					& 18    		& 6    & 24 	 \\
  \raisebox{1.5ex}[0pt]{\var{gender\_\hspace{0.3mm}sim}}		&	\resp{female\_\hspace{0.3mm}sim} 	& 17    		& 7 & 24    	 \\
  \cline{2-5}
  & Total	& 35 & 13 & 48
\end{tabular}
\caption{Simulation results, where any difference in promotion rates between \resp{male\_\hspace{0.3mm}sim} and \resp{female\_\hspace{0.3mm}sim} is purely due to chance.}
\label{discriminationRand1}
\end{table}

\begin{exercise} \label{sampleDifferenceInMaleAndFemaleDiscrimination}
What is the difference in promotion rates between the two simulated groups in Table~\ref{discriminationRand1}? How does this compare to the observed 29.2\% in the actual groups?\footnote{$18/24 - 17/24=0.042$ or about 4.2\% in favor of the men. This difference due to chance is much smaller than the difference observed in the actual groups.}
\end{exercise}


\textC{\pagebreak}

\subsection{Checking for independence}

We computed one possible difference under the independence model in Guided Practice~\ref{sampleDifferenceInMaleAndFemaleDiscrimination}, which represents one difference due to chance. While in this first simulation, we physically dealt out files, it is more efficient to perform this simulation using a computer. Repeating the simulation on a computer, we get another difference due to chance: -0.042. And another: 0.208. And so on until we repeat the simulation enough times that we have a good idea of what represents the \emph{distribution of differences from chance alone}. Figure~\ref{discRandDotPlot} shows a plot of the differences found from 100 simulations, where each dot represents a simulated difference between the proportions of male and female files that were recommended for promotion.

\begin{figure}[ht]
\centering
\includegraphics[width=0.85\textwidth]{ch_intro_to_data/figures/discRandDotPlot/discRandDotPlot}
\caption{A stacked dot plot of differences from 100 simulations produced under the independence model, $H_0$, where \var{gender\_\hspace{0.3mm}sim} and \var{decision} are independent. Two of the 100 simulations had a difference of at least 29.2\%, the difference observed in the study.}
\label{discRandDotPlot}
\end{figure}

Note that the distribution of these simulated differences is centered around 0. We simulated these differences assuming that the independence model was true, and under this condition, we expect the difference to be zero with some random fluctuation. We would generally be surprised to see a difference of \emph{exactly} 0: sometimes, just by chance, the difference is higher than 0, and other times it is lower than zero.

\begin{example}{How often would you observe a difference of at least 29.2\% (0.292) according to Figure~\ref{discRandDotPlot}? Often, sometimes, rarely, or never?}
It appears that a difference of at least 29.2\% due to chance alone would only happen about 2\% of the time according to Figure~\ref{discRandDotPlot}. Such a low probability indicates a rare event.
\end{example}

\textC{\newpage}

The difference of 29.2\% being a rare event suggests two possible interpretations of the results of the study:
\begin{itemize}
\setlength{\itemsep}{0mm}
\item[$H_0$] \textbf{Independence model.} Gender has no effect on promotion decision, and we observed a difference that would only happen rarely.
\item[$H_A$] \textbf{Alternative model.} Gender has an effect on promotion decision, and what we observed was actually due to equally qualified women being discriminated against in promotion decisions, which explains the large difference of 29.2\%.
\end{itemize}
Based on the simulations, we have two options. (1)~We conclude that the study results do not provide strong evidence against the independence model. That is, we do not have sufficiently strong evidence to conclude there was gender discrimination. (2)~We conclude the evidence is sufficiently strong to reject $H_0$ and assert that there was gender discrimination. When we conduct formal studies, usually we reject the notion that we just happened to observe a rare event.\footnote{This reasoning does not generally extend to anecdotal observations. Each of us observes incredibly rare events every day, events we could not possibly hope to predict. However, in the non-rigorous setting of anecdotal evidence, almost anything may appear to be a rare event, so the idea of looking for rare events in day-to-day activities is treacherous. For example, we might look at the lottery: there was only a 1 in 176 million chance that the Mega Millions numbers for the largest jackpot in history (March 30, 2012) would be (2, 4, 23, 38, 46) with a Mega ball of (23), but nonetheless those numbers came up! However, no matter what numbers had turned up, they would have had the same incredibly rare odds. That is, \emph{any set of numbers we could have observed would ultimately be incredibly rare}. This type of situation is typical of our daily lives: each possible event in itself seems incredibly rare, but if we consider every alternative, those outcomes are also incredibly rare. We should be cautious not to misinterpret such anecdotal evidence.} So in this case, we reject the independence model in favor of the alternative. That is, we are concluding the data provide strong evidence of gender discrimination against women by the supervisors.

\index{data!discrimination|)}

One field of statistics, statistical inference, is built on evaluating whether such differences are due to chance. In statistical inference, statisticians evaluate which model is most reasonable given the data. Errors do occur, just like rare events, and we might choose the wrong model. While we do not always choose correctly, statistical inference gives us tools to control and evaluate how often these errors occur. In Chapter~\ref{foundationsForInference}, we give a formal introduction to the problem of model selection. We spend the next two chapters building a foundation of probability and theory necessary to make that discussion rigorous.

% ex:spelllang=pt
